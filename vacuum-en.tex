\documentclass[12pt]{article}
\usepackage[utf8]{inputenc}			%% REMOVE THIS AFTER TRANSLATION
\usepackage[T2A]{fontenc}
\usepackage[russian]{babel}			%% REMOVE THIS AFTER TRANSLATION
%\usepackage[english]{babel}

\usepackage{amsmath,amssymb}
\usepackage[mathscr]{eucal}
\usepackage[notref]{showkeys}

\newcommand{\pl}{\partial}
\newcommand{\ol}{\overline}
\newcommand{\plr}{\partial r}
\newcommand{\tv}{\tilde{v}}
\newcommand{\ee}{\mathbf{e}}
\newcommand{\tu}{\tilde{u}}
\newcommand{\tw}{\tilde{w}}
\newcommand{\Dtph}{\Delta_{\Omega}}
\newcommand{\DD}{\mathcal{D}}
\newcommand{\HH}{\mathscr{H}}
%\newcommand{\SP}{\mathcal{S}}
\newcommand{\WW}{\mathscr{W}}
%\newcommand{\Ww}{\mathcal{S}}
\newcommand{\RE}{\mathrm{Re}}
\newcommand{\RR}{\mathbb{R}}
\newcommand{\CC}{\mathbb{C}}
\newcommand{\Sph}{\mathbb{S}}
\newcommand{\OO}{\mathcal{O}}
\newcommand{\ve}{\varepsilon}
\newcommand{\YY}{\mathrm{Y}}

\begin{document}
\begin{center}
{\Large
Quantum Hamiltonian Eigenstates\\[2mm]
for a Free Transverse Field
}\\
\vspace{0.3cm}
T.~A.~Bolokhov
\end{center}

\begin{abstract}
	We present a method for construction of an alternative set of states
	which satisfy the eigenstate functional equations for a quantum Hamiltonian operator
	of a free transverse field within the framework of the second quantization.
	Our recipe is based upon extensions of the quadratic form
	of the transverse Laplace operator which are used as a source of basis functions
	with a singularity at an isolated point in the three-dimensional space.
\end{abstract}


%%%%%%%%%%%%%%%%%%%%%%%%%%%%%%%%%%%%%%%%%%%%%%%%%%%%%%%%%%%%%%%%%%%%%%%%%%%%%%%%
%                                                                              %
%                                                                              %
%                            I N T R O D U C T I O N                           %
%                                                                              %
%                                                                              %
%%%%%%%%%%%%%%%%%%%%%%%%%%%%%%%%%%%%%%%%%%%%%%%%%%%%%%%%%%%%%%%%%%%%%%%%%%%%%%%%
\section*{Introduction}
	The second quantization approach
\cite{Dirac}, \cite{Becchi}
	has been the reference framework for constructing the quantum field theory
	since the time of its inception in the first half of the 20th century.
	Later in the development, for the purpose of practical calculations
	of the scattering matrix elements, a wide recognition was given
	to the technique of Feynman's diagrams,
        which is based on the Lagrangian formulation of the classical theory.
	Unlike this latter technique, one of the advantages of the second (canonical) quantization
	is that it provides a description of the quantum Hamiltonian operator.
	In a correctly defined quantum system the latter operator must be
	self-adjoint with respect to some Hilbert space.
	The well-known finite-dimensional examples 
\cite{BF},
\cite{Jackiw}
	show that upon renormalization and the removal of singularities
	the Hamiltonian nominee may well be a symmetric but still not a self-adjoint operator
	of a free particle on a restricted space of states.
	Such an candidate can be extended (completed) to a self-adjoint operator,
	however this procedure is ambiguous as it requires an introduction of an extension parameter
	(the dimensional transmutation phenomenon
\cite{Jackiw},
\cite{LFres}).
	A similar effect can seemingly be observed in the cases of systems of infinite number
	of harmonic oscillators.
	We shall argue that the quadratic part of the quantum Hamiltonian of a free transverse vector field
\begin{gather*}
    \HH_{0} = \int_{\RR^{3}} \bigl(-\frac{\delta}{\delta A_{k}(\vec{x})}
	P_{kj} \frac{\delta}{\delta A_{j}(\vec{x})}
	+ \Delta A_{j}(\vec{x}) A_{j}(\vec{x}) \bigr)d^{3}x, \\
    \partial_{k} A_{k} = 0,\quad
    \partial_{k} P_{kj} =0, \quad P^{2}=P, \quad P_{kj} = P_{jk},
\end{gather*}
	which appears, for example, in electrodynamics or as a result of renormalization
	of a gauge theory, is a limiting case of a self-adjoint extension of a certain
	symmetric operator defined on a restricted space of states.
	At the same time, generic self-adjoint extensions turn out to be dependent
	on an extension parameter and for that reason do not possess scale invariance.

	Due to the lack of adequate means of defining a scalar product on the space of functionals
	which describe states of the stationary picture of the quantum field theory
	we shall not make strict statements about the self-adjointness of operators.
	Instead, we shall provide a sketch of the vacuum state and its excitations for such operators (the Fock space
\cite{Fock}).
	These states will satisfy the equations for ``eigenstate'' functionals and
	form a hierarchy associated with creation and annihilation of particles.
	It is natural to demand that these equations match
	the functional equations
\begin{equation*}
    \HH_{0} \Phi_{\sigma_{n}}(A) = \Lambda_{\sigma_{n}} \Phi_{\sigma_{n}}(A) ,
\end{equation*}
	for eigenstates of Hamiltonian
$ \HH_{0} $,
	but at the same time they can be defined on a set of functionals
	which satisfy different conditions in the vicinity of the ``boundary'' values.
	For such boundary points in the configuration space
	in the vicinity of which the boundary conditions of the new functional space are set,
	one can take the field configurations with singularities behaving as
\begin{equation}
\label{Asing}
    \vec{A}(x) \sim \frac{\vec{A}_{0}}{|x|}, \quad |x| \to 0.
\end{equation}
	The self-adjoint extensions of the theory, therefore, will depend on a
	certain preferred (or {\it chosen})  point of the three-dimensional space.
	As the interactions or self-interactions are ``turned on'',
	such extensions of the Hamiltonian and the related states will most likely
	turn out to be unstable.
	They, however, may still contribute to the scattering matrix as
	intermediate states for particles interacting via a transverse field.

	For the sake of brevity, we introduce the following notations for the
	scalar and vector products
\begin{equation*}
    \vec{A}\cdot\vec{B} = A_{j}B_{j} ,\quad
	(\vec{A}\times\vec{B})_{l} = \epsilon_{ljk} A_{j} B_{k} ,
\end{equation*}
	and we always assume summation in the repeated indices.




%%%%%%%%%%%%%%%%%%%%%%%%%%%%%%%%%%%%%%%%%%%%%%%%%%%%%%%%%%%%%%%%%%%%%%%%%%%%%%%%
%                                                                              %
%                                                                              %
%             F I N I T E - D I M E N S I O N A L   E X A M P L E S            %
%                                                                              %
%                                                                              %
%%%%%%%%%%%%%%%%%%%%%%%%%%%%%%%%%%%%%%%%%%%%%%%%%%%%%%%%%%%%%%%%%%%%%%%%%%%%%%%%
\section{Finite-dimensional examples with\\
         singular interactions}
	In this part we describe finite-dimensional examples from quantum mechanics
	in order to make an attempt to generalize some of their properties
	to the infinite-dimensional case.
	Let
\begin{equation*}
    H_{\ve} = \Delta + \ve \delta(x)
	= -\frac{\partial^{2}}{\partial x_{k}^{2}}
	    + \ve \delta(x)
\end{equation*}
	be the Hamiltonian of a particle existing in the two- or three-dimensional space
	and interacting with a
$ \delta $-potential which is concentrated at the origin.
	Hamiltonian
$ H_{\ve} $
	does not have a correct definition in terms of a closed operator
	in the Hilbert space.
	One can, however, consider the action of
$ H_{\ve} $
	on the set of smooth functions decreasing towards the origin along with their derivatives.
	This action corresponds to a symmetric operator
\begin{equation*}
    H: \;\; H f(\vec{x}) = \Delta f(\vec{x}) =
	-\frac{\partial^{2}}{\partial x_{k}^{2}} f(\vec{x}) ,
\end{equation*}
	which, evidently, does not account for potential
$ \ve \delta(\vec{x}) $.
	In terms of the spherical coordinates in two-dimensional
\begin{equation*}
    \vec{x} = \vec{x}(r,\varphi)
    = \begin{pmatrix} r\cos\varphi\\
        r\sin\varphi
        \end{pmatrix}, \quad
    \begin{array}{l}
	0 \leq r,\\ 0 \leq\varphi < 2\pi
    \end{array}
\end{equation*}
	or three-dimensional space
\begin{equation}
\label{sphchange}
    \vec{x} = \vec{x}(r,\theta,\varphi)
    = \begin{pmatrix} r\cos\theta \cos\varphi\\
        r\cos\theta \sin\varphi\\
        r\sin\theta
        \end{pmatrix}, \quad
    \begin{array}{l}
	0 \leq r, \\
	0 \leq\theta\leq\pi,\\
	0 \leq\varphi < 2\pi
    \end{array}
\end{equation}
	the action of operator
$ H_{0} $
	has the following form.
	If a scalar function
$ f(\vec{x}) $
	is represented in terms of a sum of spherical harmonics
$ e^{il\varphi} $ or
$ \YY_{lm}(\theta,\varphi) $
	with coefficients depending on the radial variable,
\begin{gather*}
    f_{2}(\vec{x}) = f_{2}(\vec{x}(r,\varphi)) = \sum_{0\leq l}
    \frac{1}{\sqrt{r}} u_{l}(r)
        \frac{e^{il\varphi}}{\sqrt{2\pi}} , \\
    f_{3}(\vec{x}) = f_{3}(\vec{x}(r,\theta,\varphi)) = \sum_{0\leq |m| \leq l}
    \frac{1}{r} u_{lm}(r)
        \YY_{lm}(\theta,\varphi) , 
\end{gather*}
	then the corresponding operation
$ \Delta $
	acts as follows
\begin{gather*}
    \Delta f_{2}(\vec{x})
        = \sum_{0\leq l} \frac{1}{\sqrt{r}}T_{l-\frac{1}{2}} u_{l}(r)
	    \frac{e^{il\varphi}}{\sqrt{2\pi}} , \\
    \Delta f_{3}(\vec{x})
        = \sum_{0\leq |m| \leq l} \frac{1}{r}T_{l} u_{lm}(r)
	\YY_{lm}(\theta,\varphi) ,
\end{gather*}
	where
\begin{gather}
\label{Tl}
    T_{l} = -\frac{d^{2}}{dr^{2}} + \frac{l(l+1)}{r^{2}} ,\\
\nonumber
    T_{l}^{-1}(r,s) = \frac{1}{2l+1}\bigl(\frac{s^{l+1}}{r^{l}} \theta(r-s)
	+ \frac{r^{l+1}}{s^{l}}\theta(s-r)\bigr).
\end{gather}
	Note that because of orthonormality of the sets of spherical
	harmonics, the scalar product
\begin{equation*}
    (f,g)_{\RR^{2}} = \int_{\RR^{2}} \ol{f(\vec{x})} g(\vec{x}) \,d^{2}x ,
\quad
    (f,g)_{\RR^{3}} = \int_{\RR^{3}} \ol{f(\vec{x})} g(\vec{x}) \,d^{3}x ,
\end{equation*}
	descends to the coefficient functions
$ u(r) $
	as a scalar product on the half-axis
\begin{equation}
\label{plainprod}
    (u,v) = \int_{0}^{\infty} \ol{u(r)} v(r) \, dr .
\end{equation}
	Operators
$ T_{l-\frac{1}{2}} $ and
$ T_{l} $
	defined on the set of smooth functions vanishing at the origin
	along with their derivatives,
	are essentially self-adjoint ?({\it with respect to})? scalar product
(\ref{plainprod})
	at
$ l \geq 1 $.
	At the same time, operators
$ T_{-\frac{1}{2}} $,
$ T_{0} $
	acting on the latter set are symmetric operators with the deficiency indices
$ (1,1) $.
	Their self-adjoint extensions
$ T_{-\frac{1}{2}}^{\kappa} $,
$ T_{0}^{\kappa} $
	have continuous spectrum eigenfunctions that look like
\begin{gather*}
    u_{2\lambda}(r) = \sqrt{\lambda r} (\alpha_{2\lambda} J_{0}(\lambda r)
	+ \beta_{2\lambda} Y_{0}(\lambda r)) , \\
    u_{3\lambda}(r) = \alpha_{3\lambda} \sin \lambda r
	+ \beta_{3\lambda} \cos\lambda r , \\
    \alpha_{n\lambda} = \alpha_{n}(\lambda,\kappa), \quad
    \beta_{n\lambda} = \beta_{n}(\lambda,\kappa),
\end{gather*}
	along with, possibly, some eigenfunctions of the discrete spectrum.
	The actions of extensions
$ T_{-\frac{1}{2}}^{\kappa} $,
$ T_{0}^{\kappa} $
	match the differential operations
$ T_{-\frac{1}{2}} $ and
$ T_{0} $, correspondingly.

	Returning to Cartesian coordinates, therefore, symmetric operators
$ H $
	can be extended to self-adjoint operators
$ H_{2}^{\kappa} $, 
$ H_{3}^{\kappa} $
	defined on the set of functions satisfying the asymptotic conditions
\begin{equation}
\label{fas2}
    \lim_{r\to 0} \frac{f(\vec{x}(r))}{\ln r} = \kappa \lim_{r\to 0}\bigl(
	f(\vec{x}(r)) -\lim_{r'\to 0} \frac{f(\vec{x}(r'))}{\ln r'} \ln r
    \bigr) ,
\end{equation}
	or
\begin{equation}
\label{fas3}
    \lim_{r\to 0} rf(\vec{x}(r)) = -\kappa \lim_{r\to 0}(
	1 + r \frac{\partial}{\partial r} ) f(\vec{x}(r)) ,
\end{equation}
	at the origin
    (see Eqs. (3.43), (3.44) in
\cite{Jackiw}).
	Action of 
$ H_{2}^{\kappa} $,
$ H_{3}^{\kappa} $,
	still matches the sum of squares of the second derivatives
$ \Delta $ in the corresponding space
$ \RR^{2} $ or
$ \RR^{3} $.

	Extensions
$ H_{2}^{\kappa} $,
$ H_{3}^{\kappa} $
	depend on parameter
$ \kappa $,
	the dimension of which originates from the presence of dimensionality of operator
$ H $:
$ [H] = [x]^{-2} $.
	From physical perspective one can say that
$ H_{2}^{\kappa} $,
$ H_{3}^{\kappa} $
	appear as a result of renormalization of the respective operators
$ H_{\ve} $
	at
$ \ve \to 0 $.
	As for the presence of singular functions with asymptotics
(\ref{fas2}),
(\ref{fas3})
	at the origin in the domain, it can be traced to the
	renormalized singular interaction
$ \ve \delta(\vec{x}) $.
	In the case of a particle in two-dimensional space one
	has the phenomenon of dimensional transmutation ---
	a dimensionless parameter
$ \ve $
	is replaced with a dimensional parameter
$ \kappa $
	during renormalization.
\cite{LFres}.


	As another example one can consider operators of the type
\begin{equation}
\label{secondex}
    \Delta + \frac{\ve}{|x|^{2}} =
	-\frac{\partial^{2}}{\partial x_{k}^{2}} + \frac{\ve}{|x|^{2}},
\end{equation}
	with
$ \ve $ a dimensionless parameter.
	Such operators are closed symmetric operators at finite
$ \ve $
	in some vicinity of zero (in two-dimensional case
$ \ve $ has to be positive).
	When building a function satisfying the equations for eigenvalues
	one observes that the increase of the divergence of the ``eigenfunction'' by
$ |x|^{-2} $
	originating from the action of the potential cancels the
	divergence from the action of the Laplacian.
	Therefore, operator
(\ref{secondex})
	has an alternative basis of locally square-integrable
	``eigenfunctions'' behaving as
$ |x|^{-\eta} $
	near the origin
($ \eta = \sqrt{\ve} $ in two dimensions and
$ \eta = \frac{1}{2}(1+\sqrt{1+4\ve}) $ in three dimensions),
	that is, it allows self-adjoint extensions
	(this is not a mathematically strict explanation).
	One can show that in the limit
$ \ve \to 0 $
	these extensions continuously turn into the corresponding operators
$ H_{2}^{\kappa} $ and
$ H_{3}^{\kappa} $.




%%%%%%%%%%%%%%%%%%%%%%%%%%%%%%%%%%%%%%%%%%%%%%%%%%%%%%%%%%%%%%%%%%%%%%%%%%%%%%%%%%%%%%%%%%
%                                                                                        %
%                                                                                        %
%    T H R E E - D I M E N S I O N A L   T R A N S V E R S E   F I E L D   T H E O R Y   %
%                                                                                        %
%                                                                                        %
%%%%%%%%%%%%%%%%%%%%%%%%%%%%%%%%%%%%%%%%%%%%%%%%%%%%%%%%%%%%%%%%%%%%%%%%%%%%%%%%%%%%%%%%%%
\section{Three-dimensional transverse field theory}
	From the perspective of theory of operators in Hilbert space,
	the example of the last section shows that the restriction of the
	domain of operator
$ \Delta $
	to the set of smooth functions decreasing at the origin along with their derivatives
	leads to a symmetric operator and an ambiguity in the definition
	of the Hamiltonian of the system.
	At the same time, the interaction which disappears during renormalization
	only serves as a catalyst for that ambiguity by choosing a preferred point in the space.
	In this work we shall try to generalize the lessons of the finite-dimensional example
	to the case of field theory.
	We cannot really speak of self-adjointness of quantum-mechanical operators
	in the case of a configurational space with an infinite number of dimensions
	as in this case we do not have a possibility to define the scalar product
	on a wide enough class of functionals.
	For this reason we shall limit our consideration to the ambiguity
	in the construction of eigenvectors of such systems after renormalization,
	and shall provide an instructive example of an alternative set
	of vacuum and excited states.

	Consider the following Hamiltonian function
\begin{equation}
\label{qH3}
    H_{\ve}
    = \int_{\RR^{3}} \bigl(E_{j'}P_{j'k}^{\ve T}P_{kj}^{\ve}
	E_{j} + (\pl_{k} A_{j})^{2}
	+ \ve (A^{3}+\ldots) \bigr) d^{3}x ,\quad j,k= 1,2,3,
\end{equation}
    where
$ A_{k}^{a}(x) $,
$ E_{k}^{a}(x) $ are the fields of ({\it ?generalized?}) coordinates and their conjugate momenta
	in the three-dimensional space which satisfy the transversality conditions
\begin{equation}
\label{transAE}
    \pl_{k} A_{k}^{a} = 0, \quad 
    \pl_{k} E_{k}^{a} = 0 .
\end{equation}
	We have denoted as 
$ \ve (A^{3}+\ldots) $
	the homogeneous terms of dimension
$ [x]^{-4} $
	of higher order in coordinates
$ A_{k}^{a} $,
	as
$ P_{kj}^{\ve} $ --- the projector from the transverse
	to the covariant-transverse field projections,
\begin{equation*}
    P_{kj}^{\ve}
	= \delta_{kj} - \pl_{k} M^{-1} (\pl_{j}-\ve A_{j}),
	\quad M = (\pl_{j} - \ve A_{j}) \pl_{j} ,
\end{equation*}
    and as
$ \ve $ --- a small dimensionless parameter of the theory.
	Fields
$ A_{k}^{a}(x) $,
$ E_{k}^{a}(x) $
	also have an internal symmetry index
$ a $
	which is everywhere assumed to be summed upon.
	The action of the covariant derivative (and of all objects that contain it)
	may be non-trivial in this index
\begin{equation*}
    (\pl_{k}-A_{k})^{ab} = \pl_{k} \delta^{ab} - A_{k}^{c} t^{abc} .
\end{equation*}
	We shall be only considering the quadratic terms, for which the non-triviality
	of action in this index reduces to mere summation.
	Provided that the matrices 
$ t^{abc} $
	are orthogonal for different
$ c $,
	the components corresponding to different values of the upper index of field
$ A_{k}^{a}(x) $
	separate.

	An actual physical example of a Hamiltonian of type
(\ref{qH3})
	is given in the third chapter of book
\cite{FS}.
	Indeed, in Eq.
(2.5)
	therein the following Hamiltonian density is presented,
\begin{equation}
\label{hFS}
    h = \frac{1}{2} (E_{k}^{a})^{2} + \frac{1}{4}
	(\partial_{k}A_{j}^{a} - \partial_{j}A_{k}^{a}
	    - \ve [A_{j},A_{k}]^{a})^{2} ,
\end{equation}
	where
$ \vec{A}(\vec{x}) $ is the transverse field, and
	the conjugate momentum
$ E_{k}^{a} $
	is placed a constraint (2.41) upon,
\begin{equation}
\label{conn}
    (\partial_{k} - \ve A_{k}) E_{k} = 0 .
\end{equation}
	After splitting momentum
$ \vec{E}^{a}(\vec{x}) $
	into its longitudinal and transverse components
\begin{equation*}
    E_{k} = E_{k}^{L} + E_{k}^{T} ,\quad \partial_{k} E_{k}^{T} = 0,
\end{equation*}
	we obtain from condition
(\ref{conn})
	that
\begin{gather*}
    E_{k}^{L} = - \partial_{k} M^{-1} (\partial_{l} -\ve A_{l}) E_{l}^{T},
    \quad M = (\partial_{j} - \ve A_{j})\partial_{j} , \\
    E_{k} = \bigl(\delta_{kl}
	- \partial_{k} M^{-1} (\partial_{l} -\ve A_{l})\bigr) E_{l}^{T},
\end{gather*}
	and the Hamiltonian density
(\ref{hFS})
	after integrating by parts is transformed to the form of 
Eq.~(\ref{qH3})
\begin{align*}
    h =& \,\frac{1}{2} \bigl( (\delta_{kl}
	- \partial_{k} M^{-1} (\partial_{l} -\ve A_{l}) ) E_{l}^{T} \bigr)^{2}
	+ \frac{1}{2} \bigl( \partial_{k}A_{j}^{a} \bigr)^{2} +\\
	&+ \ve \partial_{k}A_{j}^{a} [A_{j}, A_{k}]^{a}
	+\frac{1}{2} \ve^{2} ([A_{j},A_{k}]^{a})^{2} .
\end{align*}

	During the renormalization procedure
$ \ve \to 0 $
	and the higher order terms
$ \ve (A^{3}+\ldots) $
	in Hamiltonian
(\ref{qH3})
	disappear, while projector
$ P_{kj}^{\ve} $
	turns into the orthogonal projector onto the transverse component
\begin{equation}
\label{Plim}
    P_{kj}^{\ve} \stackrel{\ve\to 0}{\rightarrow}
	P_{kj} = \delta_{kj} - \pl_{k} \pl^{-2} \pl_{j} ,\quad
    P_{kj}^{T} = P_{kj} ,\quad P_{kn} P_{nj} = P_{kj} .
\end{equation}
	However, replacement of
$ P_{kj}^{\ve} $ with $ P_{kj} $
	in the renormalized quantum Hamiltonian does not occur without leaving a trace.
        Classical Hamiltonian
(\ref{qH3})
	seems to have singularities via projector
$ P_{kj}^{\ve} $
	when 
$ A_{k}(x) $
	locally behaves as
$ |x|^{-1} $.
	Higher order homogeneous terms have the same singularity.
	In analogy to example
(\ref{secondex}),
	these two types of singularities can cancel each other,
	and in this way supply the renormalized quantum Hamiltonian
	with a domain having new boundary conditions and, correspondingly,
	different spectral properties.

	In order to see this better, let us consider the action
	of the quantum Hamiltonian operator --- which can be obtained from
	the classical function
(\ref{qH3})
	by taking
$ \ve = 0 $ ---
	upon functionals
$ \Phi(A) $,
\begin{equation}
\label{qH0}
    \HH_{0}\Phi(A) = - \int_{\RR^{3}} \frac{\delta}{\delta A_{k}(x)}
	P_{kj} \frac{\delta}{\delta A_{j}(x)} d^{3}x \,\Phi(A)
	+ Q(A) \Phi(A) .
\end{equation}
	Here
$ P_{kj} $
	is the projector
(\ref{Plim})
	onto the transverse subspace, and
$ Q(A) $
	is the quadratic form of the Laplace operator
$ \Delta $,
\begin{align}
\nonumber
    Q(A) = \int_{\RR^{3}} (\pl_{k}A_{j}(x))^{2} d^{3}x
	&= - \int_{\RR^{3}} A_{j}(x) \frac{\pl^{2}}{\pl x_{k}^{2}} A_{j}(x)
	    d^{3} x =\\
\label{QA}
	&= \int_{\RR^{3}} A_{j}(x) \Delta A_{j}(x) d^{3}x .
\end{align}

	The vacuum state and
$ n $-particle excitations of operator
$ \HH_{0} $
	can then be built from the Gaussian functional
\begin{gather}
\label{Phi0}
    \Phi_{0}(A) = \exp\{-\frac{1}{2}(A,P\Delta^{\frac{1}{2}}PA)\} ,\\
\label{Phin}
    \Phi_{\sigma_{n}}(A) = \iint
    \sigma_{n}^{j_{1}\ldots j_{n}} (\vec{x}_{1},\ldots \vec{x}_{n})
	b_{j_{1}}(\vec{x}_{1}) \ldots b_{j_{n}}(\vec{x_{n}})
    d^{3}x_{1} \ldots d^{3}x_{n} \Phi_{0}(A) ,
\end{gather}
	where
$ \sigma_{n} $ are some symmetric functions, and
$ b_{j}(\vec{x}) $ are the creation operators from the corresponding pair
\begin{equation*}
    b_{j}(\vec{x}) = P_{jk}\bigl(\frac{\delta}{\delta A_{k}(\vec{x})}
	- \Delta^{\frac{1}{2}}_{k}\vec{A}(\vec{x})\bigr) ,\quad
    a_{j}(\vec{x}) = P_{jk}\bigl(\frac{\delta}{\delta A_{k}(\vec{x})}
	+ \Delta^{\frac{1}{2}}_{k}\vec{A}(\vec{x})\bigr) ,
\end{equation*}
    	of creation and annihilation operators.
	Here quite essential is the fact that projector
$ P $
	commutes with operator
$ \Delta $,
	and hence, with an arbitrary function thereof ---
	for example with 
$ \Delta^{\frac{1}{2}} $.

	Operator
$ \HH_{0} $
	acts on functionals
$ \Phi_{\sigma_{n}} $
	non-diagonally, however, as is easy to see,
	it leaves
$ n $-particle subspaces invariant.
	For further diagonalization it is necessary to pass
	to the spectral representation of operator
$ \Delta $,
	which we shall do later in the framework of a more general approach.
	The main idea of that approach is
	to build an alternative hierarchy of ``eigenstates'' via a method
	which, in analogy to the second quantization, can be called
	the method of secondary self-adjoint extensions.



%%%%%%%%%%%%%%%%%%%%%%%%%%%%%%%%%%%%%%%%%%%%%%%%%%%%%%%%%%%%%%%%%%%%%%%%%%%%%%%%%%%%%%%%%%
%                                                                                        %
%  M E T H O D   O F   S E C O N D A R Y   S E L F - A D J O I N T   E X T E N S I O N S %
%                                                                                        %
%%%%%%%%%%%%%%%%%%%%%%%%%%%%%%%%%%%%%%%%%%%%%%%%%%%%%%%%%%%%%%%%%%%%%%%%%%%%%%%%%%%%%%%%%%
\subsection{Method of secondary self-adjoint extensions}
	In the case when the quantum Hamiltonian has the form of 
Eq.~(\ref{qH0}),
	and the closed quadratic form
$ Q(A) $
	{\it admits?} non-trivial extensions,
	a natural way of constructing an alternative set of ``eigenstates''
	of operator
$ \HH_{0} $ arises.
	A closed semi-bounded quadratic form
$ Q(A) $
	may be defined by means of a symmetric or self-adjoint operator
$ S $
	via a natural formula
\begin{equation*}
    Q(A) = (A,SA) .
\end{equation*}
	Herein domain
$ \DD_{S} $
	of operator
$ S $
	is contained in domain
$ \DD_{Q} $
	of form
$ Q $,
	and the latter, generally, differs from the former quite significantly.
	Symmetric operator
$ S $
	allows self-adjoint extensions
$ S_{\kappa} $
	one of which -- an extension {\it in line with} (or {\it \`a la}, or {\it due to}) Friedrichs
\cite{FStone} ---
	also defines form
$ Q $,
	while the rest of the extensions
	define different quadratic forms
$ Q_{\kappa} $, at least at finite deficiency indices of 
$ S $
(for general material on quadratic forms see section VIII.6 of book
\cite{RS1}).
	These quadratic forms in certain cases (in fact, in the majority of simple examples)
	are extensions of the original form
\begin{equation*}
    Q \subset Q_{\kappa} ,
\end{equation*}
	that is, the domain of 
$ Q $
	is contained in the closure of the domain of 
$ Q_{\kappa} $
\begin{equation*}
    \DD_{Q} \subset \ol{\DD}_{Q_{\kappa}} ,
\end{equation*}
	and for all vectors
$ A $ from
$ \DD_{Q} $
	the equality
\begin{equation*}
    Q(A) = Q_{\kappa}(A) ,\quad A\in \DD_{Q} ,
\end{equation*}
	is obeyed.
	In particular, 
\cite{Inv}
	shows spherically symmetric extensions of quadratic form
(\ref{QA})
\begin{equation}
\label{QkA}
        Q_{\kappa}(A) = \lim_{r\to 0}\Bigl(
    \int_{\RR^{3}\setminus B_{r}}
        \bigl(\frac{\partial A_{k}}{\partial x_{j}}\bigr)^{2} d^{3} x -
    (\frac{5}{3r}+ \frac{44}{27}\kappa) \int_{\partial B_{r}}
        |\vec{A}(\vec{x})|^{2} d^{2} s \Bigr) ,
\end{equation}
	for transverse vectors
$ \vec{A}(\vec{x}) $
	({\it? with respect to?}) in the scalar product
\begin{equation*}
    (\vec{A},\vec{B})_{\RR^{3}} = \int_{\RR^{3}}
	\ol{A_{j}(\vec{x})} B_{j}(\vec{x}) \,d^{3}x .
\end{equation*}
	By
$ B_{r} $ we have denoted a ball of radius
$ r $
	centered at any preferred point.
	For all vector fields that are regular at that point (which we shall
	further take to be the origin) the value of form
$ Q_{\kappa} $
	obviously equals the value of the form
(\ref{QA})
\begin{equation*}
        Q(A) = \int_{\RR^{3}}
        \bigl(\frac{\partial A_{k}}{\partial x_{j}}\bigr)^{2} d^{3} x .
\end{equation*}
	But the domain of form
$ Q_{\kappa} $
	also includes fields with singularities of type
(\ref{Asing})
	in their three transverse components
	of angular momentum
$ l=1 $.
	The reason for this is that for such fields
	the singularities of the order
$ r^{-1} $
	in the volume integral in
(\ref{QkA})
	get canceled by the singularities of the integral over the sphere.
	Notably, the domains of all non-trivial extensions
$ Q_{\kappa} $
	coincide and do not depend on
$ \kappa $.
	We should also add that the coefficient of the dimensional parameter
$ \kappa $
	in equation
(\ref{QkA})
	can be taken arbitrary,
	the value
$ \frac{44}{27} $
	has been chosen to conform to boundary conditions
(\ref{cTb}),
	which are introduced later.

	Next we note that we can demand that the basic relations for ``eigenfunctionals''
$ \Phi_{\sigma_{n}}(A) $
	of operator
$ \HH_{0} $
\begin{equation*}
    \HH_{0} \Phi_{\sigma_{n}}(A) = \Lambda_{\sigma_{n}} \Phi_{\sigma_{n}}(A)
\end{equation*}
	be obeyed only on the domain of quadratic form
$ Q(A) $.
	This is because singular fields of the form
(\ref{Asing})
	are inadmissible for higher order terms of the renormalized Hamiltonian
(\ref{qH3}).
	But on that domain the above relations would also be obeyed
	for a quantum operator with a form
$ Q_{\kappa}(A) $
	in place of form
$ Q(A) $.
	Form
$ Q_{\kappa}(A) $,
	after taking the square root and it substituting into a Gaussian integral
	of the type of 
(\ref{Phi0}),
	yields a radically different vacuum state and a different set of excitations,
	all of which correspond to a different operator
$ \HH_{\kappa} $.
	One can conclude that operator
$ \HH_{0} $
	is a self-adjoint extension of some symmetric operator
	which is defined on a set functionals rapidly vanishing near
	boundary vectors with singularities of type
(\ref{Asing}).
	This symmetric operator also admits other extensions 
$ \HH_{\kappa} $
	``the eigenstates'' for which are built by means of the quadratic form
$ Q_{\kappa}(A) $.
	For a more detailed study of such states let us switch to spherical coordinates
	and single out the subspace of angular momentum
$ l=1 $
	from the field variables.



%%%%%%%%%%%%%%%%%%%%%%%%%%%%%%%%%%%%%%%%%%%%%%%%%%%%%%%%%%%%%%%%%%%%%%%%%%%%%%%%
%                                                                              %
%              V E C T O R   S P H E R I C A L   H A R M O N I C S             %
%                                                                              %
%%%%%%%%%%%%%%%%%%%%%%%%%%%%%%%%%%%%%%%%%%%%%%%%%%%%%%%%%%%%%%%%%%%%%%%%%%%%%%%%
\subsection{Vector spherical harmonics and separation\\
	    of variables}
	Using  scalar spherical functions
$ \YY_{lm}(\theta,\varphi) $
	let us introduce three vector spherical harmonics (VSH)
\cite{VSH}:
\begin{align}
\label{VSH1}
    \vec{\Upsilon}_{lm} = & \frac{\vec{x}}{r} \YY_{lm} , \quad
        0 \leq l, \quad |m| \leq l, \\
    \vec{\Psi}_{lm} = & \tilde{l}^{-1} r \vec{\pl} \YY_{lm} , \quad
        1 \leq l , \quad |m| \leq l, \\
\label{VSH3}
    \vec{\Phi}_{lm} = & \tilde{l}^{-1} (\vec{x} \times \vec{\pl}) \YY_{lm},
        \quad 1 \leq l , \quad |m| \leq l ,
\end{align}
	which are functions of angular variables
$ \Omega = (\theta,\varphi) $, and
$ \tilde{l} = \sqrt{l(l+1)} $.
	These functions are mutually orthogonal in terms of integration over the sphere
	and normalized to one,
\begin{align*}
    \int_{\Sph^{2}} \overline{\vec{\Upsilon}_{lm}(\Omega)}
        \vec{\Psi}_{l'm'}(\Omega) d\Omega & = 0 ,\quad
    \int_{\Sph^{2}} \overline{\vec{\Upsilon}_{lm}(\Omega)}
        \vec{\Upsilon}_{l'm'}(\Omega) d\Omega = \delta_{ll'} \delta_{mm'} , \\
    \int_{\Sph^{2}} \overline{\vec{\Upsilon}_{lm}(\Omega)}
        \vec{\Phi}_{l'm'}(\Omega) d\Omega       & = 0 ,\quad
    \int_{\Sph^{2}} \overline{\vec{\Psi}_{lm}(\Omega)}
        \vec{\Psi}_{l'm'}(\Omega) d\Omega = \delta_{ll'} \delta_{mm'} , \\
    \int_{\Sph^{2}} \overline{\vec{\Phi}_{lm}(\Omega)}
        \vec{\Psi}_{l'm'}(\Omega) d\Omega & = 0 ,\quad
    \int_{\Sph^{2}} \overline{\vec{\Phi}_{lm}(\Omega)}
        \vec{\Phi}_{l'm'}(\Omega) d\Omega = \delta_{ll'} \delta_{mm'} .
\end{align*}
	The vector spherical harmonics enable one to uniquely represent
	a vector function
$ \vec{A}(\vec{x}) $
	in terms of the three sums
\begin{equation}
\label{fext}
    \vec{A}(\vec{x}) =
        \sum_{0\leq |m| \leq l} y_{lm}(r) \vec{\Upsilon}_{lm} +
        \sum_{l,m} \psi_{lm}(r) \vec{\Psi}_{lm} +
        \sum_{l,m} w_{lm}(r) \vec{\Phi}_{lm} .
\end{equation}
	For brevity we will from now on assume that summation in indices
$ l,m $
	is always taken in the range
$ 1 \leq l $, 
$ |m| \leq l $
	unless states otherwise explicitly.
	For each component of expansion
(\ref{fext}),
	when acted upon with operator
$ \Delta $,
	the following separation of variable takes place
\begin{equation*}
    \Delta \bigl(z(r) \vec{Z}_{lm}\bigr) =
-\frac{1}{r^{2}} \frac{\pl}{\plr} r^{2} \frac{\pl}{\plr} z(r) \vec{Z}_{lm}
        + \frac{z(r)}{r^{2}} \Dtph \vec{Z}_{lm}, \quad
            \vec{Z} = \vec{\Upsilon}, \vec{\Psi}, \vec{\Phi} .
\end{equation*}
	The action of the spherical Laplacian
$ \Dtph $
	on the VSH is non-diagonal (for
$ l \geq 1 $)
	but with normalization
(\ref{VSH1})--(\ref{VSH3})
	it turns out to be symmetric,
\begin{align*}
    \Dtph \vec{\Upsilon}_{lm} &= (2+\tilde{l}^{2}) \vec{\Upsilon}_{lm}
            - 2 \tilde{l} \vec{\Psi}_{lm} ,\\
                  \Dtph \vec{\Psi}_{lm} &= -2 \tilde{l}
\vec{\Upsilon}_{lm}
            + \tilde{l}^{2} \vec{\Psi}_{lm} ,\\
    \Dtph \vec{\Phi}_{lm} &= \tilde{l}^{2} \vec{\Phi}_{lm} .
\end{align*}
	If one imposes the condition of transversality
(\ref{transAE})
	upon a vector function
$ \vec{A}(\vec{x}) $,
	then it will be parametrized by just two sets of functions
$ u_{lm}(r) $,
$ w_{lm}(r) $
	instead of three as in 
(\ref{fext}),
\begin{equation}
\label{Atrexp}
    \vec{A}(\vec{x}) =
        \sum_{l,m} \bigl(\tilde{l}
	    \frac{u_{lm}}{r^{2}} \vec{\Upsilon}_{lm} +
        \frac{u_{lm}'}{r} \vec{\Psi}_{lm} 
    +   \frac{w_{lm}}{r} \vec{\Phi}_{lm} \bigr) .
\end{equation}
	The first two terms in the bracket are not transverse by themselves,
	but they become so when taken together,
\begin{align}
\label{treq}
    \vec{\pl} &\cdot
\bigl(\tilde{l}\frac{u_{lm}}{r^{2}}\vec{\Upsilon}_{lm}
        +\frac{u'_{lm}}{r}\vec{\Psi}_{lm}\bigr) =\\
\nonumber
    &= \tilde{l} \YY_{lm}
        \bigl( (\frac{u'_{lm}}{r^{2}}-\frac{2u_{lm}}{r^{3}})
        \frac{\vec{x}}{r}\cdot\frac{\vec{x}}{r} 
    + \frac{u_{lm}}{r^{2}} \vec{\pl}\cdot \frac{\vec{x}}{r} \bigr) 
    + \tilde{l}^{-1} u'_{lm} \vec{\pl}\cdot\vec{\pl} \YY_{lm} = 0 .
\end{align}
	For now we are assuming that functions
$ u_{lm}(r) $,
$ w_{lm}(r) $
	are smooth enough and fall off rapidly towards the origin.

	The action of the quadratic form of Laplace operator on a transverse field
$ \vec{A}(\vec{x}) $
	written in terms of new variables
$ u_{lm}(r) $,
$ w_{lm}(r) $
	takes the following form (see the corresponding equations in
\cite{Lapl})
\begin{equation*}
    \int_{\RR^{3}}\vec{A}(\vec{x})\cdot \Delta \vec{A}(\vec{x}) d^{3}x
	= \sum_{l,m}\langle u_{lm},\check{T}_{l}u_{lm}\rangle_{l}
	    + \sum_{l,m}(w_{lm},\check{T}_{l}w_{lm}) ,
\end{equation*}
	where
$ \langle \cdot , \cdot \rangle_{l} $
	is the scalar product inherited from 
$ \RR^{3} $
\begin{equation}
\label{angleprod}
    \langle u, v\rangle_{l} = \int_{0}^{\infty} \bigl(
	\ol{u'(r)}v'(r) + \frac{l(l+1)}{r^{2}} \ol{u(r)}v(r)\bigr) dr ,
    \quad u(0) = v(0) = 0,
\end{equation}
	while the radial part of the Laplace operator
$ \check{T}_{l} $
	and the scalar product
$ (\cdot,\cdot) $
	have been defined in
(\ref{Tl}) and
(\ref{plainprod}).
	A surprising fact, significantly reducing calculations, is that
	for smooth functions vanishing at the origin,
	product
(\ref{angleprod})
	can be defined as a sesquilinear form of operation
$ T_{l} $
	in scalar product
$ (\cdot,\cdot) $
\begin{equation}
\label{Tprod}
    \langle u,v\rangle_{l} = \int_{0}^{\infty} \ol{u(r)} \bigl(
	-\frac{d^{2}}{dr^{2}}v(r) + \frac{l(l+1)}{r^{2}}v(r) \bigr) dr
	= (u, T_{l}v).
\end{equation}
	In order to avoid confusion between the differential operation
$ T_{l} $
	arising from the scalar product and the radial part of Laplace operator,
	we will denote the latter as
$ \check{T}_{l} $
	from here on.

	The kinetic term of Hamiltonian
(\ref{qH0})
	can be re-written as follows,
\begin{multline}
\label{qHkin}
    - \iint_{\RR^{3}} \frac{\delta}{\delta A_{k}(\vec{x})}
	P_{kj}(\vec{x},\vec{y}) \frac{\delta}{\delta A_{j}(\vec{x})}
	\, d^{3}x\, d^{3}y =\\
    = -\iint\Bigl(
\frac{\delta}{\delta w_{l'm'}(r')} \frac{\delta w_{l'm'}(r')}{
    \delta A_{k}(\vec{x})}
+\frac{\delta}{\delta u_{l'm'}(r')}\frac{\delta u_{l'm'}(r')}{
    \delta A_{k}(\vec{x})} \Bigr) P_{kj}(\vec{x},\vec{y}) \times \\
    \times \Bigl(
\frac{\delta w_{lm}(r)}{\delta A_{j}(\vec{y})}
    \frac{\delta}{\delta w_{lm}(r)}
+\frac{\delta u_{lm}(r)}{\delta A_{j}(\vec{y})} \frac{\delta}{\delta u_{lm}(r)}
    \Bigr) dr\, dr'\, d^{3}x\, d^{3}y .
\end{multline}
	In order for projector
$ P_{kj}(\vec{x},\vec{y}) $
\begin{align*}
    P(\vec{x},\vec{y}) =& \sum_{l,m}
\bigl(\frac{\tilde{l}}{s} \vec{\Upsilon}_{lm}(\Omega)
    - \frac{\pl}{\pl s} \vec{\Psi}_{lm}(\Omega) \bigr) T^{-1}_{l}(s,r)
\bigl(\frac{\tilde{l}}{r} \ol{\vec{\Upsilon}_{lm}(\Omega')}
    + \frac{\pl}{\pl r} \ol{\vec{\Psi}_{lm}(\Omega')} \bigr) +\\
 +& \sum_{l,m} s^{-1} \vec{\Phi}_{lm}(\Omega) \delta(s-r)
	\ol{\vec{\Phi}_{lm}(\Omega')} r^{-1} ,
    \quad \vec{x} = (s,\Omega) , \quad \vec{y} = (r, \Omega')
\end{align*}
	to act as a unit operator on transverse functions,
	let us accept the following parametrization for new variables
$ (u_{lm}, w_{lm}) $
	in terms of 
$ \vec{A} $:
\begin{align}
\label{dphiA}
    w_{lm}(r) &= r\int d\Omega \, \ol{\vec{\Phi}_{lm}(\Omega)}\cdot
	\vec{A}(r,\Omega)
	= \frac{1}{r} \int d^{3}x \, \delta(r-s) \ol{\vec{\Phi}_{lm}(\Omega)} \cdot
	    \vec{A}(\vec{x}) , \\
\nonumber
    u_{lm}(r) &= \int ds \, T_{l}^{-1}(r,s) \int d\Omega \,\bigl(
	\tilde{l}\ol{\vec{\Upsilon}_{lm}(\Omega)}
	    -\frac{\pl}{\pl s}s\ol{\vec{\Psi}_{lm}(\Omega)}
	\bigr) \cdot \vec{A}(s,\Omega) =\\
\label{duA}
    &= \int d^{3}x \,
	\bigl(\frac{\tilde{l}}{s^{2}}T_{l}^{-1}(r,s)
	    \ol{\vec{\Upsilon}_{lm}(\Omega)}
+\frac{1}{s}(\frac{\pl}{\pl s}T_{l}^{-1}(r,s))\ol{\vec{\Psi}_{lm}(\Omega)}
	\bigr) \cdot \vec{A}(\vec{x}) ,
\end{align}
	where again
$ \vec{x} = \vec{x}(s,\Omega) $.
	It is not difficult to see that these expressions restore fields
$ u_{lm}(r) $,
$ w_{lm}(r) $ from
$ \vec{A}(\vec{x}) $
	expressed as 
(\ref{Atrexp}),
	while, at the same time, they get annihilated on any longitudinal component
\begin{equation*}
    \vec{A}^{L}(\vec{x}) = \sum_{0\leq |m| \leq l} \bigl(
	v'_{lm}(r)\vec{\Upsilon}_{lm}(\Omega) +\frac{\tilde{l}}{r}v_{lm}(r)
	    \vec{\Psi}_{lm}(\Omega)\bigr) .
\end{equation*}
	Let us calculate the variations
$ \frac{\delta w_{lm}}{\delta A} $,
$ \frac{\delta u_{lm}}{\delta A} $,
	from
(\ref{dphiA}),
(\ref{duA}) and substitute them into
(\ref{qHkin}).
	We find,
\begin{align*}
    -\int& dr\, dr'\, d^{3}x \Bigl(\frac{\delta}{\delta w_{l'm'}(r')}
\vec{\Phi}_{l'm'}(\Omega) \frac{\delta(r'-s)}{r'} \cdot
	\frac{\delta(s-r)}{r} \ol{\vec{\Phi}_{lm}(\Omega)}
	    \frac{\delta}{\delta w_{lm}(r)}
    +\\
&\quad +\frac{\delta}{\delta u_{l'm'}(r')}
    \bigl(\frac{\tilde{l}}{s^{2}} T_{l'}^{-1}(r',s)
	\vec{\Upsilon}_{l'm'}(\Omega)
    +\frac{1}{s}(\frac{\pl}{\pl s}T_{l'}^{-1}(r',s))\vec{\Psi}_{l'm'}(\Omega)
	\bigr)\cdot \\
&\quad\quad \cdot \bigl(\frac{\tilde{l}}{s^{2}}T_{l}^{-1}(s,r)
    \ol{\vec{\Upsilon}_{lm}(\Omega)} +\frac{1}{s}(\frac{\pl}{\pl s}
    T_{l}^{-1}(s,r))\ol{\vec{\Psi}_{lm}(\Omega)} \bigr)
	\frac{\delta}{\delta u_{lm}(r)} \Bigr) =\\
&= -\int dr\, \frac{\delta}{\delta w_{lm}(r)} \frac{\pl}{\pl w_{lm}(r)} -\\
&\quad    -\int dr'\, ds\, dr'\, \frac{\delta}{\delta u_{lm}(r')}
    T_{l}^{-1}(r',s)
    (-\frac{\pl^{2}}{\pl s^{2}}+\frac{\tilde{l}^{2}}{s^{2}})
    T_{l}^{-1}(s,r) \frac{\delta}{\delta u_{lm}(r)} ,
\end{align*}
	where we have immediately dropped the cross terms of
$ u $ and
$ w $
	which vanish due to orthogonality of the VSH.
	In the last term the action of
$ T_{l}^{-1} $ on
$ T_{l} $
	produces a
$ \delta $-function
	which removes one integration.
	It is worth noting here that the appearance of coefficient
$ T_{l}^{-1}(r,s) $
	in the square of conjugate ``momenta''
	({\it i.e.} in the kinetic part of the Hamiltonian) is quite natural
	if the ``coordinate'' variable 
$ u_{l}(r) $
	is measured by scalar product
(\ref{Tprod})
	involving operation
$ T_{l} $.

	Adding up the kinetic and potential parts we find the following
	expression for Hamiltonian
(\ref{qH0})
	in terms of the new variables,
\begin{align*}
    \HH_{0} =& \sum_{l,m} \bigl( -\int_{0}^{\infty} dr
	\frac{\delta}{\delta w_{lm}(r)} \frac{\delta}{\delta w_{lm}(r)}
	    + (w_{lm},\check{T}_{l} w_{lm})\bigr)+\\
    &+ \sum_{l,m} \bigl(-\iint_{0}^{\infty} dr dr'
    \frac{\delta}{\delta u_{lm}(r')} T_{l}^{-1}(r',r)
	\frac{\delta}{\delta u_{lm}(r)}
	    + \langle u_{lm}, \check{T}_{l} u_{lm}\rangle_{l} \bigr).
\end{align*}
	As was expected, variables
$ w_{lm} $, 
$ u_{lm} $
	separate for all
$ l $ and $ m $,
	while the vacuum states and excitations of this Hamiltonian
	can be sought as products of states of Hamiltonians
\begin{equation}
\label{Hlm}
    \HH_{lm} = -\iint_{0}^{\infty} dr dr'
    \frac{\delta}{\delta u_{lm}(r')} T_{l}^{-1}(r',r)
	\frac{\delta}{\delta u_{lm}(r)}
	+ \langle u_{lm}, \check{T}_{l}u_{lm}\rangle_{l}
\end{equation}
	and
\begin{equation*}
    \HH_{lm}' = -\int_{0}^{\infty} dr
	\frac{\delta}{\delta w_{lm}(r)} \frac{\delta}{\delta w_{lm}(r)}
	    + (w_{lm},\check{T}_{l} w_{lm}) .
\end{equation*}
	Hamiltonians
$ \HH_{lm}' $
	are operators found after switching to spherical coordinates
	and separating the variables {\it in a Hamiltonian for / starting from ?} a free scalar field {\it with?}
$ l \geq 1 $.
	Their eigenvectors are evidently defined unambiguously,
	and so we do not consider them in detail,
	paying close attention to operators
$ \HH_{lm} $ instead.



%%%%%%%%%%%%%%%%%%%%%%%%%%%%%%%%%%%%%%%%%%%%%%%%%%%%%%%%%%%%%%%%%%%%%%%%%%%%%%%%
%                                                                              %
%                                                                              %
%              E X T E N S I O N S   O F   O P E R A T O R   T _ 1             %
%                                                                              %
%                                                                              %
%%%%%%%%%%%%%%%%%%%%%%%%%%%%%%%%%%%%%%%%%%%%%%%%%%%%%%%%%%%%%%%%%%%%%%%%%%%%%%%%
\subsection{Extensions of quadratic form of operator $ \check{T}_{1} $}
	It was shown in
\cite{Inv}
	that operators
$ \check{T}_{1} $
	in scalar product
$ \langle \cdot , \cdot \rangle_{1} $
	are symmetric operators with deficiency indices of
$ (1,1) $.
	Such operatos have non-trivial self-adjoint extensions
	that act as mixed expressions
\begin{equation*}
    \check{T}_{\kappa} u = T u - \frac{2}{r} u'(0)
    = -\frac{d^{2}u}{dr^{2}} + \frac{2}{r^{2}} u -\frac{2}{r}u'(0) .
    \qquad\text{!!!! точка лишняя}
\end{equation*}
	on the domains
\begin{equation}
\label{cTb}
    \DD^{\kappa} = \{u(r): \quad \langle u,u\rangle_{1} < \infty, \;
	\langle \check{T}_{\kappa} u, \check{T}_{\kappa} u\rangle_{1} <\infty,
	\; 3u''(0) = 4u'(0) \} .
\end{equation}
	For brevity we will omit index
$ l=1 $
	of operator
$ \check{T}_{1\kappa} $
	and differential operation
$ T_{1} $
	both here and in the following section.
	Operators
$ \check{T}_{\kappa} $
	have a simple continuous spectrum which occupies the non-negative half-axis,
	and to which the following ``eigenfunctions'' (the kernel{\it\small s??} of the spectral transformation)
	correspond
\begin{equation*}
\label{Tpl}
    p_{\lambda}(r) = p_{1,\lambda}^{\kappa}(r)
        = \frac{2r}{\sqrt{2\pi}\lambda^{2}} \frac{d}{dr}\frac{1}{r}
    (\cos(\zeta +\lambda r) - \cos\zeta) ,
\end{equation*}
	where
\begin{equation*}
    e^{2i\zeta} = \frac{\lambda - i\kappa}{\lambda + i\kappa}.
\end{equation*}
	At
$ \kappa < 0 $
	operator
$ \check{T}_{\kappa} $
	has a simple eigenvalue
$ -\kappa $
	(the discrete spectrum) and an eigenfunction
\begin{equation*}
    q(r) = q_{\kappa}(r)
    = \sqrt{-\frac{2}{\kappa^{3}}}
        \bigl(\kappa e^{\kappa r} + \frac{1-e^{\kappa r}}{r}\bigr) .
\end{equation*}
	The set
$ \{p_{\lambda}, q \} $
	enjoys the conditions of orthogonality
\begin{equation*}
    \langle p_{\lambda} , p_{\mu} \rangle_{1} = \delta(\lambda-\mu) ,
    \quad \langle p_{\lambda} , q \rangle_{1} = 0 ,
    \quad \langle q , q \rangle_{1} = 1
\end{equation*}
	and completeness,
\begin{equation*}
    \int_{0}^{\infty} p_{\lambda}(r) T_{s} p_{\lambda}(s) \,d\lambda
        + q(r) T_{s} q(s)\bigr|_{\kappa < 0} = \delta(r-s) ,
\end{equation*}
	where index
$ s $
	of differential operation
$ T_{s} $
	emphasizes that the latter acts with respect to variable
$ s $.
	Operators
$ \check{T}_{\kappa} $
	generate extensions
$ \langle u, \check{T}_{\kappa} u\rangle_{1} $
	of the quadratic forms from the potential parts of Hamiltonians
(\ref{Hlm}).
	The original form
$ \langle u, \check{T} u\rangle_{1} $
	is defined on the set of doubly differentiable functions
	vanishing at the origin along with their first derivative
\begin{equation*}
    \WW^{2}_{0} = \{u(r):\quad \langle u,u\rangle_{1} <\infty, \;
	\langle u, \check{T}u\rangle_{1} < \infty, \; u(0)=u'(0) =0 \} ,
\end{equation*}
	which corresponds to singly differentiable fields
\begin{equation}
\label{Atrns}
    \vec{A}(\vec{x}) =
        \sqrt{2}
	    \frac{u_{1m}(r)}{r^{2}} \vec{\Upsilon}_{1m}(\theta,\varphi) +
        \frac{u_{1m}'(r)}{r} \vec{\Psi}_{1m}(\theta,\varphi)
\end{equation}
	regular at the origin.
	The extended forms
$ \langle u, \check{T}_{\kappa} u\rangle_{1} $
	are defined on a set of functions with an arbitrary bounded value
	of the derivative at the origin
\begin{equation*}
    \WW^{2}_{1} = \{u(r):\quad \langle u,u\rangle_{1} <\infty, \;
	\langle u, \check{T}_{\kappa}u\rangle_{1} < \infty, \; u(0)=0 \} .
\end{equation*}
	Obviously the following equality is obeyed on set
$ \WW^{2}_{0} $
\begin{equation*}
    \langle u, \check{T}_{\kappa} u\rangle_{1}= \langle u,
	\check{T} u\rangle_{1} ,\quad u \in \WW^{2}_{0} .
\end{equation*}
	We shall not provide a symmetric {\it limiting?} expression for the extended
	form, as we pass right on to the spectral expansion instead,
\begin{equation*}
    \langle u, \check{T}_{\kappa} u\rangle_{1}
    = \iint_{0}^{\infty} Q_{\kappa}(r,s) T_{r}u(r) T_{s}u(s) \,dr\,ds ,
\end{equation*}
	with
\begin{equation*}
    Q_{\kappa}(r,s) = \int_{0}^{\infty} p_{\lambda}(r)
    	_{\lambda}(s) \lambda^{2} \,d\lambda
        - \kappa^{2} T_{r} q(r) T_{s} q(s) \bigr|_{\kappa <0} ,
\end{equation*}
	wherein the second term exists only for
$ \kappa < 0 $.

	To conclude this subsection we note that form
$ \langle u, \check{T} u\rangle_{1} $
	is in fact a special case of form
$ \langle u, \check{T}_{\kappa} u\rangle_{1} $
	corresponding to
$ \kappa = \infty $.
	From the perspective of the spectral properties of these forms,
	one can observe that the spherical Bessel function
\begin{equation*}
    p_{1,\lambda}(r) = \frac{2r}{\sqrt{2\pi}\lambda^{2}}
	\frac{d}{dr}\frac{1}{r} \sin \lambda r ,
\end{equation*}
	appearing in the parametrization of the non-singular transverse field
(\ref{Atrns}),
	is a limiting case of function
$ p_{1,\lambda}^{\kappa} $
\begin{equation*}
    p_{1,\lambda}(r)  
        = \lim_{\kappa\to\infty}
    \frac{2r}{\sqrt{2\pi}\lambda^{2}} \frac{d}{dr}\frac{1}{r}
    (\cos(\zeta +\lambda r) - \cos\zeta) ,\quad \zeta\to -\frac{\pi}{2}.
\end{equation*}



\subsection{Hamiltonian eigenstates at
$ l=1 $}
\emph{\small В заголовках формулы надо выделять $\backslash$boldmath, а иначе они получаются тоньше чем текст!!!}
	The spectral expansion that we have obtained in the previous subsection
	now allows us to re-write the Gaussian functional
$ \phi_{0}^{\kappa}(u) $
	for the extended quantum operator
\begin{equation*}
    \HH_{1m}^{\kappa} = -\iint_{0}^{\infty} dr ds
    \frac{\delta}{\delta u(s)} T_{1}^{-1}(s,r)
	\frac{\delta}{\delta u(r)}
	+ \langle u, \check{T}_{\kappa}u\rangle_{1}
\end{equation*}
	as an exponent of an integral operator
\begin{equation*}
    \phi_{0}^{\kappa}(u)
	= \exp\{-\frac{1}{2} \iint Q_{\kappa}^{\frac{1}{2}}(r,s)
	T_{r}u(r) T_{s}u(s) \, dr\,ds\} ,
\end{equation*}
	where
\begin{equation*}
    Q_{\kappa}^{\frac{1}{2}}(r,s) = \int p_{\lambda}(r) p_{\lambda}(s)
	\lambda\,d\lambda - i\kappa q(r)q(s) \bigr|_{\kappa <0} .
\end{equation*}
	In this expression we have purposedly pulled out differential operations
$ T_{r} $, 
$ T_{s} $
	in order to obtain a more smooth kernel
$ Q_{\kappa}^{\frac{1}{2}} $.
	It is not difficult to see that functional
$ \phi_{0}^{\kappa} $
	satisfies the equation
\begin{equation*}
    \HH_{1m}^{\kappa} \phi_{0}^{\kappa}(u)
	= \Lambda_{0}^{\kappa} \phi_{0}^{\kappa}(u) ,\quad
    \Lambda_{0}^{\kappa} = \int_{0}^{\infty} T_{r}
	Q_{\kappa}^{\frac{1}{2}}(r,r') |_{r=r'} dr
\end{equation*}
	with some infinite eigenvalue
$ \Lambda_{0}^{\kappa} $.
	The creation and annihilation operators can be
	written in the following form using kernel
$ Q_{\kappa}^{\frac{1}{2}} $
\begin{align*}
    b(r) &= \int_{0}^{\infty} \bigl( Q_{\kappa}^{\frac{1}{2}}(r,s)T_{s}u(s)
	- T_{1}^{-1}(r,s) \frac{\delta}{\delta u(r)} \bigr) ds , \\
    a(r) &= \frac{\delta}{\delta u(r)} + \int_{0}^{\infty} 
	T_{r} Q_{\kappa}^{\frac{1}{2}}(r,s) T_{s} u(s) ,
\end{align*}
	while the invariant
$ n $-particle states can be built with the help of the formulas analogous to
(\ref{Phin})
\begin{equation*}
    \phi_{\sigma_{n}}(u) = \iint
    \sigma_{n} (r_{1},\ldots r_{n})
	b(r_{1}) \ldots b(r_{n})
    dr_{1} \ldots dr_{n} \, \phi_{0}(u) .
\end{equation*}
	In order to diagonalize operator
$ \HH_{1m}^{\kappa} $
	let us pass to the spectral representation of the quadratic form,
	{\it i.e.} perform the substitution
\begin{equation*}
    \hat{u}(\lambda) = \int_{0}^{\infty} p_{\lambda}(r) T u(r)\,dr, \quad
	\hat{u}_{d} =  \int_{0}^{\infty} q(r) T u(r) \bigr|_{\kappa<0}
\end{equation*}
	(note that all functions here are real), then
\begin{equation*}
    \HH_{1m}^{\kappa} = \int \bigl(
-\frac{\delta}{\delta \hat{u}(\lambda)} \frac{\delta}{\delta \hat{u}(\lambda)}
	+ \lambda^{2} \hat{u}^{2}(\lambda) \bigr)d\lambda
	- \kappa^{2} \hat{u}_{d}^{2} \bigr|_{\kappa <0} .
\end{equation*}
	This quantum Hamiltonian is associated to the following
	creation and annihilation operators
\begin{equation*}
    \hat{b}(\lambda) = \lambda \hat{u}(\lambda)
	- \frac{\delta}{\delta \hat{u}(\lambda)} ,\quad
    \hat{a}(\lambda) = \lambda \hat{u}(\lambda) 
	+ \frac{\delta}{\delta \hat{u}(\lambda)}
\end{equation*}
	and to a vacuum state
\begin{equation*}
    \hat{\phi}_{0}(\hat{u}) = \phi_{0}(u(\hat{u})) = \exp\{-\frac{1}{2}
	\int_{0}^{\infty} \hat{u}^{2}(\lambda) \lambda \,d\lambda
	+\frac{i\kappa}{2} \hat{u}_{d}^{2} \bigr|_{\kappa < 0}\} .
\end{equation*}
$ n $-particle eigenstates are constructed as intergrals with Bose-Einstein
	coefficients
$ \sigma(\lambda_{1},\ldots \sigma_{\lambda_{n}}) $
\begin{equation}
\label{hatphi}
    \hat{\phi}_{\sigma_{n}}(\hat{u}) = \iint
    \sigma_{n} (\lambda_{1},\ldots \lambda_{n}) \,
	\hat{b}(\lambda_{1}) \ldots \hat{b}(\lambda_{n}) \,
    d\lambda_{1} \ldots d\lambda_{n} \hat{\phi}_{0}(\hat{u}) ,
\end{equation}
	and, furthermore, for
$ \kappa < 0 $
	there are states related to the excitations of the discrete spectrum.



\subsection{Собственные состояния квантового гамильтониана свободного
поперечного поля}
    Собственные состояния квантового гамильтониана
(\ref{qH0}),
    в котором в место квадратичной формы
$ Q(A) $
    участвует ее расширение
(\ref{QkA}),
    строятся как произведения собственных состояний операторов
$ \HH_{lm}' $, 
$ 1\leq l, |m| \leq l $
$ \HH_{lm} $,
$ 2\leq l, |m| \leq l $
    и
$ \HH_{1m}^{\kappa} $.
    Для диагонализации первых двух наборов операторов можно использовать
    стандартное спектральное преобразование
\begin{equation*}
    \hat{u}_{lm}(\lambda)
	= \int_{0}^{\infty} p_{l,\lambda}(r) T_{l} u_{lm}(r)\,dr ,
    \quad \hat{w}_{lm}(\lambda)
	= \int_{0}^{\infty} \lambda p_{l,\lambda}(r) w_{lm}(r)\,dr ,
\end{equation*}
    где 
$ p_{l,\lambda}(r) $ --- это разновидность сферических функций Бесселя
\begin{equation*}
    p_{l,\lambda}(r) = \frac{2r^{l}}{\sqrt{2\pi}\lambda^{l+1}}
	\bigl(\frac{d}{dr}\frac{1}{r}\bigr)^{l} \sin \lambda r .
\end{equation*}
    Соответствующие операторы рождения и уничтожения, а также
    вакуумные состояния выглядят следующим образом
\begin{gather*}
    \hat{b}_{lm}(\lambda) = \lambda \hat{u}_{lm}(\lambda)
	- \frac{\delta}{\delta \hat{u}_{lm}(\lambda)} ,\quad
    \hat{a}_{lm}(\lambda) = \lambda \hat{u}_{lm}(\lambda) 
	+ \frac{\delta}{\delta \hat{u}_{lm}(\lambda)} \\
    \hat{b}'_{lm}(\lambda) = \lambda \hat{w}_{lm}(\lambda)
	- \frac{\delta}{\delta \hat{w}_{lm}(\lambda)} ,\quad
    \hat{a}'_{lm}(\lambda) = \lambda \hat{w}_{lm}(\lambda) 
	+ \frac{\delta}{\delta \hat{w}_{lm}(\lambda)} ,\\
    \hat{\phi}_{0}(\hat{u}_{lm}) = \exp\{-\frac{1}{2}
	\int_{0}^{\infty} \hat{u}_{lm}^{2}(\lambda) \lambda \,d\lambda \} ,\\
    \hat{\phi}'_{0}(\hat{w}_{lm}) = \exp\{-\frac{1}{2}
	\int_{0}^{\infty} \hat{w}_{lm}^{2}(\lambda) \lambda \,d\lambda \} .
\end{gather*}
    Диагонализация гамильтониана
$ \HH_{1m}^{\kappa} $
    с помощью преобразования
\begin{equation*}
    \hat{u}_{1m}(\lambda) = \int_{0}^{\infty} p_{1,\lambda}^{\kappa}(r)
	T_{1} u_{1m}(r)\,dr, 
\end{equation*}
    была описана в предыдущей части. Здесь стоит отметить, что в сферически
    несимметричном случае коэффициенты
$ \kappa $
    могут быть разными для компонент, соответствующим разным значениям
$ m $
    проекции момента вращения на третью ось координат.

    В результате, в переменных
$ \hat{u}_{lm} $,
$ \hat{w}_{lm} $
    получаем гамильтониан
\begin{equation*}
    \hat{\HH} = \sum_{-1\leq m\leq 1} \hat{\HH}_{1m}^{\kappa}
	+ \sum_{2\leq l, |m|\leq l} \hat{\HH}_{lm}
	+ \sum_{1\leq l, |m|\leq l} \hat{\HH}'_{lm} ,
\end{equation*}
    вакуумное состояние
\begin{equation*}
    \Phi_{0}^{\kappa} = \prod_{-1\leq m \leq 1} \phi_{1m}(\hat{u}_{1m}) \times
    \prod_{2\leq l, |m|\leq l} \phi_{lm}(\hat{u}_{lm}) \times
	\prod_{l,m} \phi'_{lm}(\hat{w}_{lm}) ,
\end{equation*}
    а
$ n $-частичные состояния получаются из формулы
(\ref{hatphi})
    с помощью замены оператора рождения
$ \hat{b}(\lambda) $ на
$ c(\lambda) $,
    который может принимать любое из значений
$ \hat{b}_{lm}(\lambda) $, $ \hat{b}'_{lm}(\lambda) $
\begin{equation*}
    \hat{\Phi}_{\sigma_{n}}(\hat{u}) = \iint
    \sigma_{n} (\lambda_{1},\ldots \lambda_{n}) \,
	c(\lambda_{1}) \ldots c(\lambda_{n}) \,
    d\lambda_{1} \ldots d\lambda_{n} \hat{\phi}_{0}(\hat{u}) .
\end{equation*}

\section{Заключение и обсуждения}
    Мы построили систему наборов состояний, удовлетворяющих собственным
    уравнениям для квантового оператора Гамильтона свободного поперечного
    поля.
    Полученные наборы в бщем случае
    зависят от выделенной точки пространства и не обладают
    масштабной инвариантностью (зависят от размерного параметра).
    Построение существенным образом использовало свойства расширений
    квадратичной формы оператора Лапласа, входящей в потенциальное слагаемое
    гамильтониана.
    Эти расширения могут быть записаны в ``инвариантной'' форме
(\ref{QkA}),
    которая, также как и условие поперечности, не подразумевает переход
    к сферическим координатам и использование какой-либо выделенной
    функциональной параметризации типа
(\ref{Atrexp}).
    Здесь возникает естественный вопрос о возможности обобщения формы
(\ref{QkA})
    на случай двух или нескольких выделенных точек пространства
\begin{equation*}
        Q_{\{\kappa\}}(A) = \lim_{r\to 0}\Bigl(
    \int_{\RR^{3}\setminus \{B_{r,n}\}}
        \bigl(\frac{\partial A_{k}}{\partial x_{j}}\bigr)^{2} d^{3} x -
    \sum_{n=1}^{N}\bigl(\frac{5}{3r}	+ \kappa_{n}\bigr)
	\int_{\partial B_{r,n}} |\vec{A}(\vec{x})|^{2} d^{2} s \Bigr) ,
\end{equation*}
    --- удовлетворяет ли такая форма условиям теоремы VIII.15 из
\cite{RS1},
    соответствует ли ей самосопряженный оператор, можно ли вычислить его
    спектральное представление. Существенная трудность при этом
    может заключаться в замыкании условия поперечности.
    Для случая одной особой точки мы просто предъявляем поперечное спектральное
    представление и далее можем рассматривать связанную с ним физику,
    а для случая нескольких точек такого представления может вообще
    не существовать.

    Другим важным замечанием является, является то, что по-видимому
    представление физического объекта (поля взаимодействия) в виде векторной
    функции на трехмерном пространстве не является правильным способом
    описания задачи. Две функции с сингулярностями в разных точках,
    которые могут являться представлениями одного и того же физического
    объекта в разные моменты времени, не выражаются через общий базис,
    то есть не имеют общего предстваления через один ортогональный набор.
    И, таким образом, имеется существенное препятствие в описании возможной
    динамики системы.

\section*{Благодарности}
    Работа выполнена при частичной поддержке грантов РФФИ 14-01-00341,
    15-01-03148 и программы ``Математические проблемы нелинейной динамики''
    РАН.

%\newpage
\begin{thebibliography}{0}

\bibitem{Dirac}
P.~A.~M.~Dirac,
``Quantum theory of emission and absorption of radiation,''
Proc.\ Roy.\ Soc.\ Lond.\ A {\bf 114} (1927) 243.

\bibitem{Becchi}
C.~M.~Becchi,
``Second quantization'', doi:10.4249/scholarpedia.7902,
http://www.scholarpedia.org/article/Second\_quantization

\bibitem{BF}
    F.~A.~Berezin, L.~D.~Faddeev,
  ``A Remark on Schrodinger's equation with a singular potential,''
  Sov.\ Math.\ Dokl.\  {\bf 2} (1961) 372
  [Dokl.\ Akad.\ Nauk Ser.\ Fiz.\  {\bf 137} (1961) 1011].

\bibitem{Jackiw}
  R.~Jackiw,
  ``Delta function potentials in two-dimensional and three-dimensional
  quantum mechanics,''
  In *Jackiw, R.: Diverse topics in theoretical and mathematical physics*
  35-53 (1991).

\bibitem{LFres}
L.~D.~Faddeev,
``Notes on divergences and dimensional transmutation in Yang-Mills theory,''
Theor.\ Math.\ Phys.\  {\bf 148} (2006) 986
[Teor.\ Mat.\ Fiz.\  {\bf 148} (2006) 133].

\bibitem{Fock}
V.~Fock, ``Konfigurationsraum und zweite Quantelung,''
Z. Phys. {\bf 75} (1932), 622-647.

\bibitem{FS}
  L.~D.~Faddeev and A.~A.~Slavnov,
\emph{Gauge Fields. Introduction To Quantum Theory},
Front.\ Phys.\  {\bf 50} (1980) 1, [Front.\ Phys.\  {\bf 83} (1990) 1].

\bibitem{Lapl} T.~A.~Bolokhov,
``Extensions of the quadratic form of the transverse Laplace operator'',
arXiv:1410.1487 [math.SP].

\bibitem{FStone}
    K.~Friedrichs, ``Spektraltheorie halbbeschr\"ankter Operatoren,''
    Math. Ann. {\bf 109}, 1934, 465--487;\\
    M.~Stone, in \emph{Linear Transformations in Hilbert spaces and their
    Applications in Analysis}, Amer. Math. Soc. Colloquim Publication {\bf 15},
    Providence, R.I., 1932;\\
    or see theorem X.23 in \cite{RS2}.

\bibitem{RS1}  M.~Reed, B.~Simon, \emph{Methods of Modern Mathematical
Physics. 1. Functional Analysis}, Academic Press New York London, 1972.

\bibitem{RS2} M.~Reed, B.~Simon, \emph{Methods of Modern Mathematical Physics.
II: Fourier Analysis, Self-adjointness}, Academic Press, 1975.

\bibitem{Inv} T.~A.~Bolokhov,
``Properties of the l=1 radial part of the Laplace operator in a special
scalar product'', arXiv:1510.07824 [math.SP].
    
\bibitem{VSH}
    B.~F.~Schutz, \emph{Geometrical methods of mathematical physics},
    Cambridge University Press, 1982;\\
    E.~L.~Hill, ``The Theory of Vector Spherical Harmonics'',
    Am. J. Phys. {\bf 22} (1954) 211.

%\bibitem{Krein}
%    M.~G.~Krein, ``The theory of self-adjoint extensions of semi-bounded
%Hermitian transformations and its applications.'',
%    Rec. Math. (Mat. Sbornik) N.S., {\bf 20} (62), 1947, 431--495;
%    Rec. Math. (Mat. Sbornik) N.S., {\bf 21} (64), 1947, 365--404.

%\bibitem{F}
%    Faddeev, Clay

%\bibitem{AK}
%    S.~Albeverio, P.~Kurasov, \emph{Singular Perturbation of Differential
%    Operators. Solvable Schr\"odinger type Operators},
%    Cambridge University Press, 2000.

%\bibitem{Richt}
%    R.~D.~Richtmyer, \emph{Principles of Advanced Mathematical Physics, vol.1},
%    Springer-Verlag, New York Heildelberg Berlin, 1978.
    

\end{thebibliography}

\end{document}

