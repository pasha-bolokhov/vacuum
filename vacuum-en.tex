\documentclass[12pt]{article}
\usepackage[utf8]{inputenc}			%% REMOVE THIS AFTER TRANSLATION
\usepackage[T2A]{fontenc}
\usepackage[russian]{babel}			%% REMOVE THIS AFTER TRANSLATION
%\usepackage[english]{babel}

\usepackage{amsmath,amssymb}
\usepackage[mathscr]{eucal}
\usepackage[notref]{showkeys}

\newcommand{\pl}{\partial}
\newcommand{\ol}{\overline}
\newcommand{\plr}{\partial r}
\newcommand{\tv}{\tilde{v}}
\newcommand{\ee}{\mathbf{e}}
\newcommand{\tu}{\tilde{u}}
\newcommand{\tw}{\tilde{w}}
\newcommand{\Dtph}{\Delta_{\Omega}}
\newcommand{\DD}{\mathcal{D}}
\newcommand{\HH}{\mathscr{H}}
%\newcommand{\SP}{\mathcal{S}}
\newcommand{\WW}{\mathscr{W}}
%\newcommand{\Ww}{\mathcal{S}}
\newcommand{\RE}{\mathrm{Re}}
\newcommand{\RR}{\mathbb{R}}
\newcommand{\CC}{\mathbb{C}}
\newcommand{\Sph}{\mathbb{S}}
\newcommand{\OO}{\mathcal{O}}
\newcommand{\ve}{\varepsilon}
\newcommand{\YY}{\mathrm{Y}}

\begin{document}
\begin{center}
{\Large
Quantum Hamiltonian Eigenstates\\[2mm]
for a Free Transverse Field
}\\
\vspace{0.3cm}
T.~A.~Bolokhov
\end{center}

\begin{abstract}
	We present a method for construction of an alternative set of states
	which satisfy the eigenstate functional equations for a quantum Hamiltonian operator
	of a free transverse field within the framework of the second quantization.
	Our recipe is based upon extensions of the quadratic form
	of the transverse Laplace operator which are used as a source of basis functions
	with a singularity at an isolated point in the three-dimensional space.
\end{abstract}


%%%%%%%%%%%%%%%%%%%%%%%%%%%%%%%%%%%%%%%%%%%%%%%%%%%%%%%%%%%%%%%%%%%%%%%%%%%%%%%%
%                                                                              %
%                                                                              %
%                            I N T R O D U C T I O N                           %
%                                                                              %
%                                                                              %
%%%%%%%%%%%%%%%%%%%%%%%%%%%%%%%%%%%%%%%%%%%%%%%%%%%%%%%%%%%%%%%%%%%%%%%%%%%%%%%%
\section*{Introduction}
	The second quantization approach
\cite{Dirac}, \cite{Becchi}
	has been the reference framework for constructing the quantum field theory
	since the time of its inception in the first half of the 20th century.
	Later in the development, for the purpose of practical calculations
	of the scattering matrix elements, a wide recognition was given
	to the technique of Feynman's diagrams,
        which is based on the Lagrangian formulation of the classical theory.
	Unlike this latter technique, one of the advantages of the second (canonical) quantization
	is that it provides a description of the quantum Hamiltonian operator.
	In a correctly defined quantum system the latter operator must be
	self-adjoint with respect to some Hilbert space.
	The well-known finite-dimensional examples 
\cite{BF},
\cite{Jackiw}
	show that upon renormalization and the removal of singularities
	the Hamiltonian nominee may well be a symmetric but still not a self-adjoint operator
	of a free particle on a restricted space of states.
	Such an candidate can be extended (completed) to a self-adjoint operator,
	however this procedure is ambiguous as it requires an introduction of an extension parameter
	(the dimensional transmutation phenomenon
\cite{Jackiw},
\cite{LFres}).
	A similar effect can seemingly be observed in the cases of systems of infinite number
	of harmonic oscillators.
	We shall argue that the quadratic part of the quantum Hamiltonian of a free transverse vector field
\begin{gather*}
    \HH_{0} = \int_{\RR^{3}} \bigl(-\frac{\delta}{\delta A_{k}(\vec{x})}
	P_{kj} \frac{\delta}{\delta A_{j}(\vec{x})}
	+ \Delta A_{j}(\vec{x}) A_{j}(\vec{x}) \bigr)d^{3}x, \\
    \partial_{k} A_{k} = 0,\quad
    \partial_{k} P_{kj} =0, \quad P^{2}=P, \quad P_{kj} = P_{jk},
\end{gather*}
	which appears, for example, in electrodynamics or as a result of renormalization
	of a gauge theory, is a limiting case of a self-adjoint extension of a certain
	symmetric operator defined on a restricted space of states.
	At the same time, generic self-adjoint extensions turn out to be dependent
	on an extension parameter and for that reason do not possess scale invariance.

	Due to the lack of adequate means of defining a scalar product on the space of functionals
	which describe states of the stationary picture of the quantum field theory
	we shall not make strict statements about the self-adjointness of operators.
	Instead, we shall provide a sketch of the vacuum state and its excitations for such operators (the Fock space
\cite{Fock}).
	These states will satisfy the equations for ``eigenstate'' functionals and
	form a hierarchy associated with creation and annihilation of particles.
	It is natural to demand that these equations match
	the functional equations
\begin{equation*}
    \HH_{0} \Phi_{\sigma_{n}}(A) = \Lambda_{\sigma_{n}} \Phi_{\sigma_{n}}(A) ,
\end{equation*}
	for eigenstates of Hamiltonian
$ \HH_{0} $,
	but at the same time they can be defined on a set of functionals
	which satisfy different conditions in the vicinity of the ``boundary'' values.
	For such boundary points in the configuration space
	in the vicinity of which the boundary conditions of the new functional space are set,
	one can take the field configurations with singularities behaving as
\begin{equation}
\label{Asing}
    \vec{A}(x) \sim \frac{\vec{A}_{0}}{|x|}, \quad |x| \to 0.
\end{equation}
	The self-adjoint extensions of the theory, therefore, will depend on a
	certain preferred (or {\it chosen})  point of the three-dimensional space.
	As the interactions or self-interactions are ``turned on'',
	such extensions of the Hamiltonian and the related states will most likely
	turn out to be unstable.
	They, however, may still contribute to the scattering matrix as
	intermediate states for particles interacting via a transverse field.

	For the sake of brevity, we introduce the following notations for the
	scalar and vector products
\begin{equation*}
    \vec{A}\cdot\vec{B} = A_{j}B_{j} ,\quad
	(\vec{A}\times\vec{B})_{l} = \epsilon_{ljk} A_{j} B_{k} ,
\end{equation*}
	and we always assume summation in the repeated indices.




%%%%%%%%%%%%%%%%%%%%%%%%%%%%%%%%%%%%%%%%%%%%%%%%%%%%%%%%%%%%%%%%%%%%%%%%%%%%%%%%
%                                                                              %
%                                                                              %
%             F I N I T E - D I M E N S I O N A L   E X A M P L E S            %
%                                                                              %
%                                                                              %
%%%%%%%%%%%%%%%%%%%%%%%%%%%%%%%%%%%%%%%%%%%%%%%%%%%%%%%%%%%%%%%%%%%%%%%%%%%%%%%%
\section{Finite-dimensional examples with\\
         singular interactions}
	In this part we describe finite-dimensional examples from quantum mechanics
	in order to make an attempt to generalize some of their properties
	to the infinite-dimensional case.
	Let
\begin{equation*}
    H_{\ve} = \Delta + \ve \delta(x)
	= -\frac{\partial^{2}}{\partial x_{k}^{2}}
	    + \ve \delta(x)
\end{equation*}
	be the Hamiltonian of a particle existing in the two- or three-dimensional space
	and interacting with a
$ \delta $-potential which is concentrated at the origin.
	Hamiltonian
$ H_{\ve} $
	does not have a correct definition in terms of a closed operator
	in the Hilbert space.
	One can, however, consider the action of
$ H_{\ve} $
	on the set of smooth functions decreasing towards the origin along with their derivatives.
	This action corresponds to a symmetric operator
\begin{equation*}
    H: \;\; H f(\vec{x}) = \Delta f(\vec{x}) =
	-\frac{\partial^{2}}{\partial x_{k}^{2}} f(\vec{x}) ,
\end{equation*}
	which, evidently, does not account for potential
$ \ve \delta(\vec{x}) $.
	In terms of the spherical coordinates in two-dimensional
\begin{equation*}
    \vec{x} = \vec{x}(r,\varphi)
    = \begin{pmatrix} r\cos\varphi\\
        r\sin\varphi
        \end{pmatrix}, \quad
    \begin{array}{l}
	0 \leq r,\\ 0 \leq\varphi < 2\pi
    \end{array}
\end{equation*}
	or three-dimensional space
\begin{equation}
\label{sphchange}
    \vec{x} = \vec{x}(r,\theta,\varphi)
    = \begin{pmatrix} r\cos\theta \cos\varphi\\
        r\cos\theta \sin\varphi\\
        r\sin\theta
        \end{pmatrix}, \quad
    \begin{array}{l}
	0 \leq r, \\
	0 \leq\theta\leq\pi,\\
	0 \leq\varphi < 2\pi
    \end{array}
\end{equation}
	the action of operator
$ H_{0} $
	has the following form.
	If a scalar function
$ f(\vec{x}) $
	is represented in terms of a sum of spherical harmonics
$ e^{il\varphi} $ or
$ \YY_{lm}(\theta,\varphi) $
	with coefficients depending on the radial variable,
\begin{gather*}
    f_{2}(\vec{x}) = f_{2}(\vec{x}(r,\varphi)) = \sum_{0\leq l}
    \frac{1}{\sqrt{r}} u_{l}(r)
        \frac{e^{il\varphi}}{\sqrt{2\pi}} , \\
    f_{3}(\vec{x}) = f_{3}(\vec{x}(r,\theta,\varphi)) = \sum_{0\leq |m| \leq l}
    \frac{1}{r} u_{lm}(r)
        \YY_{lm}(\theta,\varphi) , 
\end{gather*}
	then the corresponding operation
$ \Delta $
	acts as follows
\begin{gather*}
    \Delta f_{2}(\vec{x})
        = \sum_{0\leq l} \frac{1}{\sqrt{r}}T_{l-\frac{1}{2}} u_{l}(r)
	    \frac{e^{il\varphi}}{\sqrt{2\pi}} , \\
    \Delta f_{3}(\vec{x})
        = \sum_{0\leq |m| \leq l} \frac{1}{r}T_{l} u_{lm}(r)
	\YY_{lm}(\theta,\varphi) ,
\end{gather*}
	where
\begin{gather}
\label{Tl}
    T_{l} = -\frac{d^{2}}{dr^{2}} + \frac{l(l+1)}{r^{2}} ,\\
\nonumber
    T_{l}^{-1}(r,s) = \frac{1}{2l+1}\bigl(\frac{s^{l+1}}{r^{l}} \theta(r-s)
	+ \frac{r^{l+1}}{s^{l}}\theta(s-r)\bigr).
\end{gather}
	Note that because of orthonormality of the sets of spherical
	harmonics, the scalar product
\begin{equation*}
    (f,g)_{\RR^{2}} = \int_{\RR^{2}} \ol{f(\vec{x})} g(\vec{x}) \,d^{2}x ,
\quad
    (f,g)_{\RR^{3}} = \int_{\RR^{3}} \ol{f(\vec{x})} g(\vec{x}) \,d^{3}x ,
\end{equation*}
	descends to the coefficient functions
$ u(r) $
	as a scalar product on the semi-axis
\begin{equation}
\label{plainprod}
    (u,v) = \int_{0}^{\infty} \ol{u(r)} v(r) \, dr .
\end{equation}
	Operators
$ T_{l-\frac{1}{2}} $ and
$ T_{l} $
	defined on the set of smooth functions vanishing at the origin
	along with their derivatives,
	are essentially self-adjoint ?({\it with respect to})? scalar product
(\ref{plainprod})
	at
$ l \geq 1 $.
	At the same time, operators
$ T_{-\frac{1}{2}} $,
$ T_{0} $
	acting on the latter set are symmetric operators with the deficiency indices
$ (1,1) $.
	Their self-adjoint extensions
$ T_{-\frac{1}{2}}^{\kappa} $,
$ T_{0}^{\kappa} $
	have continuous spectrum eigenfunctions that look like
\begin{gather*}
    u_{2\lambda}(r) = \sqrt{\lambda r} (\alpha_{2\lambda} J_{0}(\lambda r)
	+ \beta_{2\lambda} Y_{0}(\lambda r)) , \\
    u_{3\lambda}(r) = \alpha_{3\lambda} \sin \lambda r
	+ \beta_{3\lambda} \cos\lambda r , \\
    \alpha_{n\lambda} = \alpha_{n}(\lambda,\kappa), \quad
    \beta_{n\lambda} = \beta_{n}(\lambda,\kappa),
\end{gather*}
	along with, possibly, some eigenfunctions of the discrete spectrum.
	The actions of extensions
$ T_{-\frac{1}{2}}^{\kappa} $,
$ T_{0}^{\kappa} $
	match the differential operations
$ T_{-\frac{1}{2}} $ and
$ T_{0} $, correspondingly.

	Returning to Cartesian coordinates, therefore, symmetric operators
$ H $
	can be extended to self-adjoint operators
$ H_{2}^{\kappa} $, 
$ H_{3}^{\kappa} $
	defined on the set of functions satisfying the asymptotic conditions
\begin{equation}
\label{fas2}
    \lim_{r\to 0} \frac{f(\vec{x}(r))}{\ln r} = \kappa \lim_{r\to 0}\bigl(
	f(\vec{x}(r)) -\lim_{r'\to 0} \frac{f(\vec{x}(r'))}{\ln r'} \ln r
    \bigr) ,
\end{equation}
	or
\begin{equation}
\label{fas3}
    \lim_{r\to 0} rf(\vec{x}(r)) = -\kappa \lim_{r\to 0}(
	1 + r \frac{\partial}{\partial r} ) f(\vec{x}(r)) ,
\end{equation}
	at the origin
    (see Eqs. (3.43), (3.44) in
\cite{Jackiw}).
	Action of 
$ H_{2}^{\kappa} $,
$ H_{3}^{\kappa} $,
	still matches the sum of squares of the second derivatives
$ \Delta $ in the corresponding space
$ \RR^{2} $ or
$ \RR^{3} $.

	Extensions
$ H_{2}^{\kappa} $,
$ H_{3}^{\kappa} $
	depend on parameter
$ \kappa $,
	the dimension of which originates from the presence of dimensionality of operator
$ H $:
$ [H] = [x]^{-2} $.
	From physical perspective one can say that
$ H_{2}^{\kappa} $,
$ H_{3}^{\kappa} $
	appear as a result of renormalization of the respective operators
$ H_{\ve} $
	at
$ \ve \to 0 $.
	As for the presence of singular functions with asymptotics
(\ref{fas2}),
(\ref{fas3})
	at the origin in the domain, it can be traced to the
	renormalized singular interaction
$ \ve \delta(\vec{x}) $.
	In the case of a particle in two-dimensional space one
	has the phenomenon of dimensional transmutation ---
	a dimensionless parameter
$ \ve $
	is replaced with a dimensional parameter
$ \kappa $
	during renormalization.
\cite{LFres}.


	As another example one can consider operators of the type
\begin{equation}
\label{secondex}
    \Delta + \frac{\ve}{|x|^{2}} =
	-\frac{\partial^{2}}{\partial x_{k}^{2}} + \frac{\ve}{|x|^{2}},
\end{equation}
	with
$ \ve $ a dimensionless parameter.
	Such operators are closed symmetric operators at finite
$ \ve $
	in some vicinity of zero (in two-dimensional case
$ \ve $ has to be positive).
	When building a function satisfying the equations for eigenvalues
	one observes that the increase of the divergence of the ``eigenfunction'' by
$ |x|^{-2} $
	originating from the action of the potential cancels the
	divergence from the action of the Laplacian.
	Therefore, operator
(\ref{secondex})
	has an alternative basis of locally square-integrable
	``eigenfunctions'' behaving as
$ |x|^{-\eta} $
	near the origin
($ \eta = \sqrt{\ve} $ in two dimensions and
$ \eta = \frac{1}{2}(1+\sqrt{1+4\ve}) $ in three dimensions),
	that is, it allows self-adjoint extensions
	(this is not a mathematically strict explanation).
	One can show that in the limit
$ \ve \to 0 $
	these extensions continuously turn into the corresponding operators
$ H_{2}^{\kappa} $ and
$ H_{3}^{\kappa} $.




%%%%%%%%%%%%%%%%%%%%%%%%%%%%%%%%%%%%%%%%%%%%%%%%%%%%%%%%%%%%%%%%%%%%%%%%%%%%%%%%%%%%%%%%%%
%                                                                                        %
%                                                                                        %
%    T H R E E - D I M E N S I O N A L   T R A N S V E R S E   F I E L D   T H E O R Y   %
%                                                                                        %
%                                                                                        %
%%%%%%%%%%%%%%%%%%%%%%%%%%%%%%%%%%%%%%%%%%%%%%%%%%%%%%%%%%%%%%%%%%%%%%%%%%%%%%%%%%%%%%%%%%
\section{Three-dimensional transverse field theory}
	From the perspective of theory of operators in Hilbert space,
	the example of the last section shows that the restriction of the
	domain of operator
$ \Delta $
	to the set of smooth functions decreasing at the origin along with their derivatives
	leads to a symmetric operator and an ambiguity in the definition
	of the Hamiltonian of the system.
    А взаимодействие, которое исчезает в результате перенормировки,
    служит лишь катализатором этой неоднозначности, так как оно выделяет
    точку в пространстве.
    В этой работе мы попытаемся обобщить уроки из конечномерного примера
    на случай теории поля.
    Мы не можем говорить о самосопряженности квантово-механических
    операторов в случае конфигурационного пространства с бесконечным
    числом измерений, так как в этом случае
    у нас нет возможности определить скалярное произведение на достаточно
    широком классе функционалов.
    Поэтому мы ограничимся рассмотрением
    неоднозначности в построении собственных векторов таких систем
    после перенормировки и приведем содержательный пример альтернативного
    набора из вакуумного и возбужденных состояний.

    Рассмотрим следующую функцию Гамильтона
\begin{equation}
\label{qH3}
    H_{\ve}
    = \int_{\RR^{3}} \bigl(E_{j'}P_{j'k}^{\ve T}P_{kj}^{\ve}
	E_{j} + (\pl_{k} A_{j})^{2}
	+ \ve (A^{3}+\ldots) \bigr) d^{3}x ,\quad j,k= 1,2,3,
\end{equation}
    где
$ A_{k}^{a}(x) $,
$ E_{k}^{a}(x) $ -- поля координат и сопряженных к ним импульсов
    в трехмерном пространстве,
    удовлетворяющие условиям поперечности
\begin{equation}
\label{transAE}
    \pl_{k} A_{k}^{a} = 0, \quad 
    \pl_{k} E_{k}^{a} = 0 .
\end{equation}
    Через
$ \ve (A^{3}+\ldots) $
    обозначены однородные слагаемые старших порядков по координатам
$ A_{k}^{a} $
    рамерности
$ [x]^{-4} $,
$ P_{kj}^{\ve} $ --- проектор с поперечной
    на ковариантно-поперечную проекцию поля
\begin{equation*}
    P_{kj}^{\ve}
	= \delta_{kj} - \pl_{k} M^{-1} (\pl_{j}-\ve A_{j}),
	\quad M = (\pl_{j} - \ve A_{j}) \pl_{j} ,
\end{equation*}
    а
$ \ve $ --- малый безразмерный параметр теории.
    Поля
$ A_{k}^{a}(x) $,
$ E_{k}^{a}(x) $
    имеют также индекс внутренней симметрии
$ a $,
    по которому везде подразумевается суммирование. Действие ковариантной
    производной (и всех объектов, которые ее включают)
    может быть нетривиальными по этому индексу
\begin{equation*}
    (\pl_{k}-A_{k})^{ab} = \pl_{k} \delta^{ab} - A_{k}^{c} t^{abc} .
\end{equation*}
    Далее мы будем рассматривать только квадратичные
    слагаемые, в которых нетривиальность действия по этому индексу
    низводится до суммирования. При условии ортогональности матриц
$ t^{abc} $
    при разных
$ c $,
    компоненты, соответствующие разным значениям верхнего индекса у поля
$ A_{k}^{a}(x) $,
    разделяются.

    Реальный физический пример гамильтониана типа
(\ref{qH3})
    приведен в третьей главе книги
\cite{FS}.
    В самом деле, в формуле
(2.5)
    приводится следующая плотность гамильтониана
\begin{equation}
\label{hFS}
    h = \frac{1}{2} (E_{k}^{a})^{2} + \frac{1}{4}
	(\partial_{k}A_{j}^{a} - \partial_{j}A_{k}^{a}
	    - \ve [A_{j},A_{k}]^{a})^{2} ,
\end{equation}
    где
$ \vec{A}(\vec{x}) $ --- поперечное поле, а на сопряженный импульс
$ E_{k}^{a} $
    наложено условие связи (2.41)
\begin{equation}
\label{conn}
    (\partial_{k} - \ve A_{k}) E_{k} = 0 .
\end{equation}
    После разделения импульса
$ \vec{E}^{a}(\vec{x}) $
    на продольную и поперечную компоненты
\begin{equation*}
    E_{k} = E_{k}^{L} + E_{k}^{T} ,\quad \partial_{k} E_{k}^{T} = 0,
\end{equation*}
    из условия
(\ref{conn})
    получаем, что
\begin{gather*}
    E_{k}^{L} = - \partial_{k} M^{-1} (\partial_{l} -\ve A_{l}) E_{l}^{T},
    \quad M = (\partial_{j} - \ve A_{j})\partial_{j} , \\
    E_{k} = \bigl(\delta_{kl}
	- \partial_{k} M^{-1} (\partial_{l} -\ve A_{l})\bigr) E_{l}^{T},
\end{gather*}
    и плотность гамильтониана
(\ref{hFS}),
    с учетом интегрирования по частям, преобразуется к виду
(\ref{qH3})
\begin{align*}
    h =& \,\frac{1}{2} \bigl( (\delta_{kl}
	- \partial_{k} M^{-1} (\partial_{l} -\ve A_{l}) ) E_{l}^{T} \bigr)^{2}
	+ \frac{1}{2} \bigl( \partial_{k}A_{j}^{a} \bigr)^{2} +\\
	&+ \ve \partial_{k}A_{j}^{a} [A_{j}, A_{k}]^{a}
	+\frac{1}{2} \ve^{2} ([A_{j},A_{k}]^{a})^{2} .
\end{align*}

    При проведении перенормировки
$ \ve \to 0 $
    в гамильтониане
(\ref{qH3})
    слагаемые старших порядков
$ \ve (A^{3}+\ldots) $
    пропадают, а проектор
$ P_{kj}^{\ve} $
    переходит в ортогональный проектор на поперечную составляющую
\begin{equation}
\label{Plim}
    P_{kj}^{\ve} \stackrel{\ve\to 0}{\rightarrow}
	P_{kj} = \delta_{kj} - \pl_{k} \pl^{-2} \pl_{j} ,\quad
    P_{kj}^{T} = P_{kj} ,\quad P_{kn} P_{nj} = P_{kj} .
\end{equation}
    Однако замена
$ P_{kj}^{\ve} $ на $ P_{kj} $
    в перенормированном квантовом гамильтониане не проходит бесследно.
    Классический гамильтониан
(\ref{qH3})
    через проектор
$ P_{kj}^{\ve} $
    по-видимому имеет особенности когда
$ A_{k}(x) $
    ведет себя локально как
$ |x|^{-1} $.
    Такую же особенность имеют и однородные слагаемые старших порядков.
    По аналогии с примером
(\ref{secondex}),
    эти два типа особенностей могут сокращаться и, тем самым, давать
    для перенормированного квантового гамильтониана область
    определения с новыми граничными условиями и, соответственно, с другими
    спектральными свойствами.

    Для того, чтобы это увидеть, рассмотрим действие квантового оператора
    Гамильтона, который получается из классической функции
(\ref{qH3})
    при
$ \ve = 0 $,
    на функциналы
$ \Phi(A) $
\begin{equation}
\label{qH0}
    \HH_{0}\Phi(A) = - \int_{\RR^{3}} \frac{\delta}{\delta A_{k}(x)}
	P_{kj} \frac{\delta}{\delta A_{j}(x)} d^{3}x \,\Phi(A)
	+ Q(A) \Phi(A) ,
\end{equation}
    здесь
$ P_{kj} $ ---
    это проектор
(\ref{Plim}) на поперечное подпространство, а
$ Q(A) $ -- это квадратичная форма оператора Лапласа
$ \Delta $
\begin{align}
\nonumber
    Q(A) = \int_{\RR^{3}} (\pl_{k}A_{j}(x))^{2} d^{3}x
	&= - \int_{\RR^{3}} A_{j}(x) \frac{\pl^{2}}{\pl x_{k}^{2}} A_{j}(x)
	    d^{3} x =\\
\label{QA}
	&= \int_{\RR^{3}} A_{j}(x) \Delta A_{j}(x) d^{3}x .
\end{align}

    Вакуумное состояние и
$ n $-частичные возбуждения оператора
$ \HH_{0} $
    далее строятся на основе гауссова функционала
\begin{gather}
\label{Phi0}
    \Phi_{0}(A) = \exp\{-\frac{1}{2}(A,P\Delta^{\frac{1}{2}}PA)\} ,\\
\label{Phin}
    \Phi_{\sigma_{n}}(A) = \iint
    \sigma_{n}^{j_{1}\ldots j_{n}} (\vec{x}_{1},\ldots \vec{x}_{n})
	b_{j_{1}}(\vec{x}_{1}) \ldots b_{j_{n}}(\vec{x_{n}})
    d^{3}x_{1} \ldots d^{3}x_{n} \Phi_{0}(A) ,
\end{gather}
    где
$ \sigma_{n} $ -- симметричные функции, а
\begin{equation*}
    b_{j}(\vec{x}) = P_{jk}\bigl(\frac{\delta}{\delta A_{k}(\vec{x})}
	- \Delta^{\frac{1}{2}}_{k}\vec{A}(\vec{x})\bigr) ,\quad
    a_{j}(\vec{x}) = P_{jk}\bigl(\frac{\delta}{\delta A_{k}(\vec{x})}
	+ \Delta^{\frac{1}{2}}_{k}\vec{A}(\vec{x})\bigr) ,
\end{equation*}
    -- соответствующие операторы рождения и уничтожения.
    При этом существенную роль играет тот факт, что проектор
$ P $
    коммутирует с оператором
$ \Delta $,
    а значит и с любой его функцией, например, с 
$ \Delta^{\frac{1}{2}} $.

    Действие оператора
$ \HH_{0} $
    на функционалы
$ \Phi_{\sigma_{n}} $
    недиагонально, однако, как несложно увидеть, оно оставляет инвариантными
$ n $-частичные подпространства.
    Для дальнейшей диагонализации необходимо перейти к спектральному
    представлению оператора
$ \Delta $,
    что мы сделаем позже в рамках обобщенного подхода.
    Основная его идея заключается в построении альтернативной
    иерархии ``собственных'' состояний с помощью метода, который, по аналогии
    с методом вторичного квантования, может быть назван методом вторичных
    самосопряженных расширений.

\subsection{Метод вторичных самосопряженных расширений}
    В случае, когда квантовый гамильтониан имеет вид
(\ref{qH0}),
    а замкнутая квадратичная форма
$ Q(A) $
    имеет нетривиальные расширения, возникает естественный способ
    построения альтернативного набор ``собственных'' состояний оператора
$ \HH_{0} $.
    Замкнутая полуограниченная квадратичная форма
$ Q(A) $
    может быть задана с помощью замкнутого симметрического или самосопряженного
    оператора
$ S $
    с помощью естественной формулы
\begin{equation*}
    Q(A) = (A,SA) .
\end{equation*}
    При этом область определения 
$ \DD_{S} $
    оператора
$ S $
    содержится в области определения 
$ \DD_{Q} $
    формы
$ Q $,
    а последняя, вообще говоря, существенно отличаться от
$ \DD_{S} $.
    Симметрический оператор
$ S $
    имеет самосопряженные расширения
$ S_{\kappa} $,
    одно из которых, расширение по Фридрихсу
\cite{FStone},
    также задает форму
$ Q $,
    а остальные, по крайней мере при конечных индексах дефекта
$ S $,
    задают другие квадратичные формы
$ Q_{\kappa} $
(общие сведения о квадратичных формах см. в разделе VIII.6 книги
\cite{RS1}).
    Эти квадратичные формы в некоторых случаях (в большинстве простых примеров)
    являются расширениями исходной формы
\begin{equation*}
    Q \subset Q_{\kappa} ,
\end{equation*}
    то есть область определения
$ Q $
    содержится в замыкании области определения
$ Q_{\kappa} $
\begin{equation*}
    \DD_{Q} \subset \ol{\DD}_{Q_{\kappa}} ,
\end{equation*}
    и для всех векторов
$ A $ из
$ \DD_{Q} $
    выполняется равентсво
\begin{equation*}
    Q(A) = Q_{\kappa}(A) ,\quad A\in \DD_{Q}.
\end{equation*}
    В частности, в работе
\cite{Inv}
    приведены сферически симметричные расширения квадратичной формы
(\ref{QA})
\begin{equation}
\label{QkA}
        Q_{\kappa}(A) = \lim_{r\to 0}\Bigl(
    \int_{\RR^{3}\setminus B_{r}}
        \bigl(\frac{\partial A_{k}}{\partial x_{j}}\bigr)^{2} d^{3} x -
    (\frac{5}{3r}+ \frac{44}{27}\kappa) \int_{\partial B_{r}}
        |\vec{A}(\vec{x})|^{2} d^{2} s \Bigr) ,
\end{equation}
    для поперечных векторов
$ \vec{A}(\vec{x}) $
    в скалярном произведении
\begin{equation*}
    (\vec{A},\vec{B})_{\RR^{3}} = \int_{\RR^{3}}
	\ol{A_{j}(\vec{x})} B_{j}(\vec{x}) \,d^{3}x .
\end{equation*}
    Здесь
$ B_{r} $ -- это шар радиуса
$ r $
    с центром в какой-либо выделенной точке.
    Для всех векторных полей, регулярных в этой точке (далее это будет
    начало координат), значение
    формы
$ Q_{\kappa} $,
    очевидно, совпадает со значением формы
(\ref{QA})
\begin{equation*}
        Q(A) = \int_{\RR^{3}}
        \bigl(\frac{\partial A_{k}}{\partial x_{j}}\bigr)^{2} d^{3} x .
\end{equation*}
    Но в область определения формы
$ Q_{\kappa} $
    также попадают поля с сингулярностями вида
(\ref{Asing})
    у трех поперечных компонет со значением момента вращения
$ l=1 $.
    Это происходит потому, что для таких полей сингулярности порядка
$ r^{-1} $
    объемного интеграла в
(\ref{QkA})
    сокращаются с сингулярностями от интеграла по сфере.
    При этом область определения всех нетривиальных расширений
$ Q_{\kappa} $
    одинакова и не зависит от
$ \kappa $.
    Также отметим, что в формуле
(\ref{QkA})
    коэффициент при размерном параметре
$ \kappa $
    может быть произвольным, значение
$ \frac{44}{27} $
    выбрано для согласования с граничными условиями
(\ref{cTb}),
    которые будут введены позже.

    Далее заметим, что мы можем требовать выполнения основных соотношений
    для ``собственных'' функционалов
$ \Phi_{\sigma_{n}}(A) $
    оператора
$ \HH_{0} $
\begin{equation*}
    \HH_{0} \Phi_{\sigma_{n}}(A) = \Lambda_{\sigma_{n}} \Phi_{\sigma_{n}}(A)
\end{equation*}
    только на области определения квадратичной формы
$ Q(A) $.
    Так как сингулярные поля вида
(\ref{Asing})
    являются недопустимыми для слагаемых старших порядков неперенормированного
    гамильтониана
(\ref{qH3}).
    Но на такой области эти соотношения будут выполнены и для
    квантового оператора с формой
$ Q_{\kappa}(A) $
    на месте формы
$ Q(A) $.
    Форма
$ Q_{\kappa}(A) $
    при извлечении квадратного корня и подстановке в гауссов функционал
    вида
(\ref{Phi0})
    дает принципиально другое ваккумное состояние и другой набор возбуждений,
    которые соответствуют другому оператору
$ \HH_{\kappa} $.
    Из этого можно сделать вывод, что оператор
$ \HH_{0} $
    является самосопряженным расширением некоторого симметрического оператора,
    заданного на множестве функционалов быстро убывающих вблизи
    граничных векторов с сингулярностями вида
(\ref{Asing}).
    Этот симметрический оператор имеет также другие расширения
$ \HH_{\kappa} $,
    ``собственные'' состояния которых строятся с помощью
    квадратичной формы
$ Q_{\kappa}(A) $.
    Для более подробного описания этих состояний перейдем
    к сферическим координатам и выделим из полевых переменных подпространство
    с моментом вращения
$ l=1 $.
    

\subsection{Векторные сферические гармоники и разделение переменных}
    С помощью скалярных сферических функций
$ \YY_{lm}(\theta,\varphi) $
    введем три векторные сферические гармоники VSH
\cite{VSH}:
\begin{align}
\label{VSH1}
    \vec{\Upsilon}_{lm} = & \frac{\vec{x}}{r} \YY_{lm} , \quad
        0 \leq l, \quad |m| \leq l, \\
    \vec{\Psi}_{lm} = & \tilde{l}^{-1} r \vec{\pl} \YY_{lm} , \quad
        1 \leq l , \quad |m| \leq l, \\
\label{VSH3}
    \vec{\Phi}_{lm} = & \tilde{l}^{-1} (\vec{x} \times \vec{\pl}) \YY_{lm},
        \quad 1 \leq l , \quad |m| \leq l .
\end{align}
    -- функции угловых переменных
$ \Omega = (\theta,\varphi) $, где
$ \tilde{l} = \sqrt{l(l+1)} $.
    Эти функции ортогональны друг другу при интегрировании по сфере
    и нормированы на единицу:
\begin{align*}
    \int_{\Sph^{2}} \overline{\vec{\Upsilon}_{lm}(\Omega)}
        \vec{\Psi}_{l'm'}(\Omega) d\Omega & = 0 ,\quad
    \int_{\Sph^{2}} \overline{\vec{\Upsilon}_{lm}(\Omega)}
        \vec{\Upsilon}_{l'm'}(\Omega) d\Omega = \delta_{ll'} \delta_{mm'} , \\
    \int_{\Sph^{2}} \overline{\vec{\Upsilon}_{lm}(\Omega)}
        \vec{\Phi}_{l'm'}(\Omega) d\Omega       & = 0 ,\quad
    \int_{\Sph^{2}} \overline{\vec{\Psi}_{lm}(\Omega)}
        \vec{\Psi}_{l'm'}(\Omega) d\Omega = \delta_{ll'} \delta_{mm'} , \\
    \int_{\Sph^{2}} \overline{\vec{\Phi}_{lm}(\Omega)}
        \vec{\Psi}_{l'm'}(\Omega) d\Omega & = 0 ,\quad
    \int_{\Sph^{2}} \overline{\vec{\Phi}_{lm}(\Omega)}
        \vec{\Phi}_{l'm'}(\Omega) d\Omega = \delta_{ll'} \delta_{mm'} .
\end{align*}
    Векторные сферические гармоники позволяют написать однозначное
    представление векторной функции
$ \vec{A}(\vec{x}) $
    в виде трех сумм
\begin{equation}
\label{fext}
    \vec{A}(\vec{x}) =
        \sum_{0\leq |m| \leq l} y_{lm}(r) \vec{\Upsilon}_{lm} +
        \sum_{l,m} \psi_{lm}(r) \vec{\Psi}_{lm} +
        \sum_{l,m} w_{lm}(r) \vec{\Phi}_{lm} .
\end{equation}
    Для сокращения обозначений далее мы будем подразумевать что суммирование
    по индексам
$ l,m $
    ведется в пределах
$ 1 \leq l $, 
$ |m| \leq l $,
    если не указаны другие условия.
    Для каждой компоненты разложения
(\ref{fext})
    при действии оператора
$ \Delta $
    имеет место разделение переменных
\begin{equation*}
    \Delta \bigl(z(r) \vec{Z}_{lm}\bigr) =
-\frac{1}{r^{2}} \frac{\pl}{\plr} r^{2} \frac{\pl}{\plr} z(r) \vec{Z}_{lm}
        + \frac{z(r)}{r^{2}} \Dtph \vec{Z}_{lm}, \quad
            \vec{Z} = \vec{\Upsilon}, \vec{\Psi}, \vec{\Phi} .
\end{equation*}
    Действие сферического лапласиана
$ \Dtph $
    на VSH недиагонально (при
$ l \geq 1 $), но при нормировке
(\ref{VSH1})--(\ref{VSH3})
    оказывается симметричным:
\begin{align*}
    \Dtph \vec{\Upsilon}_{lm} &= (2+\tilde{l}^{2}) \vec{\Upsilon}_{lm}
            - 2 \tilde{l} \vec{\Psi}_{lm} ,\\
                  \Dtph \vec{\Psi}_{lm} &= -2 \tilde{l}
\vec{\Upsilon}_{lm}
            + \tilde{l}^{2} \vec{\Psi}_{lm} ,\\
    \Dtph \vec{\Phi}_{lm} &= \tilde{l}^{2} \vec{\Phi}_{lm} .
\end{align*}
    Если на векторную функцию
$ \vec{A}(\vec{x}) $
    наложить условия поперечности
(\ref{transAE}),
    то она будет параметризоваться уже не тремя, как
(\ref{fext}),
    а двумя наборами функций
$ u_{lm}(r) $,
$ w_{lm}(r) $:
\begin{equation}
\label{Atrexp}
    \vec{A}(\vec{x}) =
        \sum_{l,m} \bigl(\tilde{l}
	    \frac{u_{lm}}{r^{2}} \vec{\Upsilon}_{lm} +
        \frac{u_{lm}'}{r} \vec{\Psi}_{lm} 
    +   \frac{w_{lm}}{r} \vec{\Phi}_{lm} \bigr) .
\end{equation}
    Первые два слагаемые под знаком суммы по отдельности не поперечны, однако,
    при объединении они становятся таковыми:
\begin{align}
\label{treq}
    \vec{\pl} &\cdot
\bigl(\tilde{l}\frac{u_{lm}}{r^{2}}\vec{\Upsilon}_{lm}
        +\frac{u'_{lm}}{r}\vec{\Psi}_{lm}\bigr) =\\
\nonumber
    &= \tilde{l} \YY_{lm}
        \bigl( (\frac{u'_{lm}}{r^{2}}-\frac{2u_{lm}}{r^{3}})
        \frac{\vec{x}}{r}\cdot\frac{\vec{x}}{r} 
    + \frac{u_{lm}}{r^{2}} \vec{\pl}\cdot \frac{\vec{x}}{r} \bigr) 
    + \tilde{l}^{-1} u'_{lm} \vec{\pl}\cdot\vec{\pl} \YY_{lm} = 0 .
\end{align}
    Здесь мы пока предполагаем, что функции
$ u_{lm}(r) $,
$ w_{lm}(r) $
    достаточно гладкие и быстро убывают в начале координат.

    Действие квадратичной формы оператора Лапласа на поперечное поле
$ \vec{A}(\vec{x}) $
    в терминах новых переменных
$ u_{lm}(r) $,
$ w_{lm}(r) $
    выглядит следующим образом (см. формулы в
\cite{Lapl}):
\begin{equation*}
    \int_{\RR^{3}}\vec{A}(\vec{x})\cdot \Delta \vec{A}(\vec{x}) d^{3}x
	= \sum_{l,m}\langle u_{lm},\check{T}_{l}u_{lm}\rangle_{l}
	    + \sum_{l,m}(w_{lm},\check{T}_{l}w_{lm}) ,
\end{equation*}
    где
$ \langle \cdot , \cdot \rangle_{l} $
    -- это скалярное произведение, перенесенное из пространства
$ \RR^{3} $
\begin{equation}
\label{angleprod}
    \langle u, v\rangle_{l} = \int_{0}^{\infty} \bigl(
	\ol{u'(r)}v'(r) + \frac{l(l+1)}{r^{2}} \ol{u(r)}v(r)\bigr) dr ,
    \quad u(0) = v(0) = 0,
\end{equation}
    а радиальная часть оператора Лапласа
$ \check{T}_{l} $
    и скаларное произведение
$ (\cdot,\cdot) $
    были определены в
(\ref{Tl}) и
(\ref{plainprod}).
    Удивительный факт, существенно облегчающий вычисления, состоит в том,
    что произведение
(\ref{angleprod})
    для гладких функций, убывающих в нуле может быть задано как
    полуторалинейная форма операции
$ T_{l} $
    в скалярном произведении
$ (\cdot,\cdot) $
\begin{equation}
\label{Tprod}
    \langle u,v\rangle_{l} = \int_{0}^{\infty} \ol{u(r)} \bigl(
	-\frac{d^{2}}{dr^{2}}v(r) + \frac{l(l+1)}{r^{2}}v(r) \bigr) dr
	= (u, T_{l}v).
\end{equation}
    Для того, чтобы не путать дифференциальную операцию
$ T_{l} $,
    идущую от скалярного произведения, и радиальную часть оператора
    Лапласа, здесь и далее мы будем обозначать последнюю как
$ \check{T}_{l} $.

    Кинетическое слагаемое в гамильтониане
(\ref{qH0})
    можно переписать в следующем виде
\begin{multline}
\label{qHkin}
    - \iint_{\RR^{3}} \frac{\delta}{\delta A_{k}(\vec{x})}
	P_{kj}(\vec{x},\vec{y}) \frac{\delta}{\delta A_{j}(\vec{x})}
	\, d^{3}x\, d^{3}y =\\
    = -\iint\Bigl(
\frac{\delta}{\delta w_{l'm'}(r')} \frac{\delta w_{l'm'}(r')}{
    \delta A_{k}(\vec{x})}
+\frac{\delta}{\delta u_{l'm'}(r')}\frac{\delta u_{l'm'}(r')}{
    \delta A_{k}(\vec{x})} \Bigr) P_{kj}(\vec{x},\vec{y}) \times \\
    \times \Bigl(
\frac{\delta w_{lm}(r)}{\delta A_{j}(\vec{y})}
    \frac{\delta}{\delta w_{lm}(r)}
+\frac{\delta u_{lm}(r)}{\delta A_{j}(\vec{y})} \frac{\delta}{\delta u_{lm}(r)}
    \Bigr) dr\, dr'\, d^{3}x\, d^{3}y .
\end{multline}
    Для того, чтобы проектор
$ P_{kj}(\vec{x},\vec{y}) $
\begin{align*}
    P(\vec{x},\vec{y}) =& \sum_{l,m}
\bigl(\frac{\tilde{l}}{s} \vec{\Upsilon}_{lm}(\Omega)
    - \frac{\pl}{\pl s} \vec{\Psi}_{lm}(\Omega) \bigr) T^{-1}_{l}(s,r)
\bigl(\frac{\tilde{l}}{r} \ol{\vec{\Upsilon}_{lm}(\Omega')}
    + \frac{\pl}{\pl r} \ol{\vec{\Psi}_{lm}(\Omega')} \bigr) +\\
 +& \sum_{l,m} s^{-1} \vec{\Phi}_{lm}(\Omega) \delta(s-r)
	\ol{\vec{\Phi}_{lm}(\Omega')} r^{-1} ,
    \quad \vec{x} = (s,\Omega) , \quad \vec{y} = (r, \Omega')
\end{align*}
    действовал на поперечные функции
    как единичный оператор, выберем следующую параметризацию
    новых переменных
$ (u_{lm}, w_{lm}) $
    через
$ \vec{A} $:
\begin{align}
\label{dphiA}
    w_{lm}(r) &= r\int d\Omega \, \ol{\vec{\Phi}_{lm}(\Omega)}\cdot
	\vec{A}(r,\Omega)
	= \frac{1}{r} \int d^{3}x \, \delta(r-s) \ol{\vec{\Phi}_{lm}(\Omega)} \cdot
	    \vec{A}(\vec{x}) , \\
\nonumber
    u_{lm}(r) &= \int ds \, T_{l}^{-1}(r,s) \int d\Omega \,\bigl(
	\tilde{l}\ol{\vec{\Upsilon}_{lm}(\Omega)}
	    -\frac{\pl}{\pl s}s\ol{\vec{\Psi}_{lm}(\Omega)}
	\bigr) \cdot \vec{A}(s,\Omega) =\\
\label{duA}
    &= \int d^{3}x \,
	\bigl(\frac{\tilde{l}}{s^{2}}T_{l}^{-1}(r,s)
	    \ol{\vec{\Upsilon}_{lm}(\Omega)}
+\frac{1}{s}(\frac{\pl}{\pl s}T_{l}^{-1}(r,s))\ol{\vec{\Psi}_{lm}(\Omega)}
	\bigr) \cdot \vec{A}(\vec{x}) ,
\end{align}
    где опять
$ \vec{x} = \vec{x}(s,\Omega) $.
    Несложно увидеть, что эти выражения восстанавливают поля
$ u_{lm}(r) $,
$ w_{lm}(r) $ из
$ \vec{A}(\vec{x}) $,
    представленного в виде
(\ref{Atrexp}),
    и, в то же время, они обнуляются на любой продольной составляющей
\begin{equation*}
    \vec{A}^{L}(\vec{x}) = \sum_{0\leq |m| \leq l} \bigl(
	v'_{lm}(r)\vec{\Upsilon}_{lm}(\Omega) +\frac{\tilde{l}}{r}v_{lm}(r)
	    \vec{\Psi}_{lm}(\Omega)\bigr) .
\end{equation*}
    Далее подставим вариации
$ \frac{\delta w_{lm}}{\delta A} $,
$ \frac{\delta u_{lm}}{\delta A} $,
    вычисленные из
(\ref{dphiA}),
(\ref{duA}), в
(\ref{qHkin})
    и получим
\begin{align*}
    -\int& dr\, dr'\, d^{3}x \Bigl(\frac{\delta}{\delta w_{l'm'}(r')}
\vec{\Phi}_{l'm'}(\Omega) \frac{\delta(r'-s)}{r'} \cdot
	\frac{\delta(s-r)}{r} \ol{\vec{\Phi}_{lm}(\Omega)}
	    \frac{\delta}{\delta w_{lm}(r)}
    +\\
&\quad +\frac{\delta}{\delta u_{l'm'}(r')}
    \bigl(\frac{\tilde{l}}{s^{2}} T_{l'}^{-1}(r',s)
	\vec{\Upsilon}_{l'm'}(\Omega)
    +\frac{1}{s}(\frac{\pl}{\pl s}T_{l'}^{-1}(r',s))\vec{\Psi}_{l'm'}(\Omega)
	\bigr)\cdot \\
&\quad\quad \cdot \bigl(\frac{\tilde{l}}{s^{2}}T_{l}^{-1}(s,r)
    \ol{\vec{\Upsilon}_{lm}(\Omega)} +\frac{1}{s}(\frac{\pl}{\pl s}
    T_{l}^{-1}(s,r))\ol{\vec{\Psi}_{lm}(\Omega)} \bigr)
	\frac{\delta}{\delta u_{lm}(r)} \Bigr) =\\
&= -\int dr\, \frac{\delta}{\delta w_{lm}(r)} \frac{\pl}{\pl w_{lm}(r)} -\\
&\quad    -\int dr'\, ds\, dr'\, \frac{\delta}{\delta u_{lm}(r')}
    T_{l}^{-1}(r',s)
    (-\frac{\pl^{2}}{\pl s^{2}}+\frac{\tilde{l}^{2}}{s^{2}})
    T_{l}^{-1}(s,r) \frac{\delta}{\delta u_{lm}(r)} ,
\end{align*}
    здесь мы сразу опустили перекрестные слагаемые между
$ u $ и
$ w $,
    которые зануляются ввиду ортогональности VSH.
    В последнем слагаемом действие
$ T_{l}^{-1} $ на
$ T_{l} $
    дает
$ \delta $-функцию, в результате чего одно интегрирование снимается.
    Тут стоит отметить, что появление коэффициента
$ T_{l}^{-1}(r,s) $
    при квадрате сопряженных ``импульсов'' (то есть в кинетической части
    гамильтониана) является естественным,
    если ``координатная'' переменная 
$ u_{l}(r) $
    измеряется скалярным произведением
(\ref{Tprod})
    с операцией
$ T_{l} $.

    Суммируя преобразованные кинетическую и потенциальную части
    получим следующее выражение для гамильтониана
(\ref{qH0})
    в новых переменных
\begin{align*}
    \HH_{0} =& \sum_{l,m} \bigl( -\int_{0}^{\infty} dr
	\frac{\delta}{\delta w_{lm}(r)} \frac{\delta}{\delta w_{lm}(r)}
	    + (w_{lm},\check{T}_{l} w_{lm})\bigr)+\\
    &+ \sum_{l,m} \bigl(-\iint_{0}^{\infty} dr dr'
    \frac{\delta}{\delta u_{lm}(r')} T_{l}^{-1}(r',r)
	\frac{\delta}{\delta u_{lm}(r)}
	    + \langle u_{lm}, \check{T}_{l} u_{lm}\rangle_{l} \bigr).
\end{align*}
    Как и следовало ожидать, переменные
$ w_{lm} $, 
$ u_{lm} $
    разделяются при всех
$ l $ и $ m $,
    а вакуумные состояния и возбуждения такого гамильтониана можно искать
    в виде произведений состояний гамильтонианов
\begin{equation}
\label{Hlm}
    \HH_{lm} = -\iint_{0}^{\infty} dr dr'
    \frac{\delta}{\delta u_{lm}(r')} T_{l}^{-1}(r',r)
	\frac{\delta}{\delta u_{lm}(r)}
	+ \langle u_{lm}, \check{T}_{l}u_{lm}\rangle_{l}
\end{equation}
    и
\begin{equation*}
    \HH_{lm}' = -\int_{0}^{\infty} dr
	\frac{\delta}{\delta w_{lm}(r)} \frac{\delta}{\delta w_{lm}(r)}
	    + (w_{lm},\check{T}_{l} w_{lm}) .
\end{equation*}
    Гамильтонианы
$ \HH_{lm}' $
    представляют из себя операторы, которые получаются при
    переходе к сферическим координатам и разделении переменных для
    свободного скалярного поля при
$ l \geq 1 $.
    Их собственные векторы, по-видимому определены однозначно,
    поэтому мы не будем на них останавливаться и
    далее сосредоточимся только на операторах
$ \HH_{lm} $.

\subsection{Расширения квадратичной формы оператора
$ \check{T}_{1} $}
    В работе
\cite{Inv}
    показывается, что операторы
$ \check{T}_{1} $
    в склярном произведении
$ \langle \cdot , \cdot \rangle_{1} $
    являются симметрическими операторами с индексами дефекта
$ (1,1) $.
    Такие операторы имеют нетривиальные самосопряженные расширения,
    которые действуют как смешанные выражения
\begin{equation*}
    \check{T}_{\kappa} u = T u - \frac{2}{r} u'(0)
    = -\frac{d^{2}u}{dr^{2}} + \frac{2}{r^{2}} u -\frac{2}{r}u'(0) .
\end{equation*}
    на областях определения
\begin{equation}
\label{cTb}
    \DD^{\kappa} = \{u(r): \quad \langle u,u\rangle_{1} < \infty, \;
	\langle \check{T}_{\kappa} u, \check{T}_{\kappa} u\rangle_{1} <\infty,
	\; 3u''(0) = 4u'(0) \} .
\end{equation}
    Для упрощения обозначений в этой и в следующей части
    мы будем опускать индекс
$ l=1 $
    у оператора
$ \check{T}_{1\kappa} $
    и дифференциальной операции
$ T_{1} $.
    Операторы
$ \check{T}_{\kappa} $
    имеют однократный непрерывный спектр, занимающий нетрицательную полуось,
    которому соответствуют ``собственные'' функции (ядро спектрального
    преобразования)
\begin{equation*}
\label{Tpl}
    p_{\lambda}(r) = p_{1,\lambda}^{\kappa}(r)
        = \frac{2r}{\sqrt{2\pi}\lambda^{2}} \frac{d}{dr}\frac{1}{r}
    (\cos(\zeta +\lambda r) - \cos\zeta) ,
\end{equation*}
    где
\begin{equation*}
    e^{2i\zeta} = \frac{\lambda - i\kappa}{\lambda + i\kappa}.
\end{equation*}
    При
$ \kappa < 0 $
    оператор
$ \check{T}_{\kappa} $
    имеет однократное собственное значение
$ -\kappa $
    (дискретный спектр) и собственную функцию
\begin{equation*}
    q(r) = q_{\kappa}(r)
    = \sqrt{-\frac{2}{\kappa^{3}}}
        \bigl(\kappa e^{\kappa r} + \frac{1-e^{\kappa r}}{r}\bigr) .
\end{equation*}
    Для набора
$ \{p_{\lambda}, q \} $
    выполнены условия ортогональности
\begin{equation*}
    \langle p_{\lambda} , p_{\mu} \rangle_{1} = \delta(\lambda-\mu) ,
    \quad \langle p_{\lambda} , q \rangle_{1} = 0 ,
    \quad \langle q , q \rangle_{1} = 1
\end{equation*}
    и полноты
\begin{equation*}
    \int_{0}^{\infty} p_{\lambda}(r) T_{s} p_{\lambda}(s) \,d\lambda
        + q(r) T_{s} q(s)\bigr|_{\kappa < 0} = \delta(r-s) ,
\end{equation*}
    здесь индекс
$ s $
    у дифференциальной операции
$ T_{s} $
    подчеркивает, что она действует на переменную
$ s $.
    Операторы
$ \check{T}_{\kappa} $
    порождают расширения
$ \langle u, \check{T}_{\kappa} u\rangle_{1} $
    квадратичных форм из потенциальной части гамильтонианов
(\ref{Hlm}).
    Исходная форма
$ \langle u, \check{T} u\rangle_{1} $
    задается на множестве два раза дифференцируемых функций, исчезающих
    в нуле вместе с первой производной
\begin{equation*}
    \WW^{2}_{0} = \{u(r):\quad \langle u,u\rangle_{1} <\infty, \;
	\langle u, \check{T}u\rangle_{1} < \infty, \; u(0)=u'(0) =0 \} ,
\end{equation*}
    которое соответствует один раз дифференцируемым полям
\begin{equation}
\label{Atrns}
    \vec{A}(\vec{x}) =
        \sqrt{2}
	    \frac{u_{1m}(r)}{r^{2}} \vec{\Upsilon}_{1m}(\theta,\varphi) +
        \frac{u_{1m}'(r)}{r} \vec{\Psi}_{1m}(\theta,\varphi) ,
\end{equation}
    регулярным в начале координат.
    Расширенные формы
$ \langle u, \check{T}_{\kappa} u\rangle_{1} $
    определены на множестве функций с произвольным ограниченным значением
    производной в нуле
\begin{equation*}
    \WW^{2}_{1} = \{u(r):\quad \langle u,u\rangle_{1} <\infty, \;
	\langle u, \check{T}_{\kappa}u\rangle_{1} < \infty, \; u(0)=0 \} .
\end{equation*}
    Очевидно, что на множестве
$ \WW^{2}_{0} $
    выполняется равество
\begin{equation*}
    \langle u, \check{T}_{\kappa} u\rangle_{1}= \langle u,
	\check{T} u\rangle_{1} ,\quad u \in \WW^{2}_{0} .
\end{equation*}
    Мы не будем приводить симметричное предельное выражение для расширенных
    форм, и сразу запишем спектральное разложение
\begin{equation*}
    \langle u, \check{T}_{\kappa} u\rangle_{1}
    = \iint_{0}^{\infty} Q_{\kappa}(r,s) T_{r}u(r) T_{s}u(s) \,dr\,ds ,
\end{equation*}
    где
\begin{equation*}
    Q_{\kappa}(r,s) = \int_{0}^{\infty} p_{\lambda}(r)
    	_{\lambda}(s) \lambda^{2} \,d\lambda
        - \kappa^{2} T_{r} q(r) T_{s} q(s) \bigr|_{\kappa <0} ,
\end{equation*}
    а второе слагаемое присутствует только при
$ \kappa < 0 $.

    В заключении части стоит отметить, что форма
$ \langle u, \check{T} u\rangle_{1} $
    является частным случаем формы
$ \langle u, \check{T}_{\kappa} u\rangle_{1} $,
    который соответствует значению
$ \kappa = \infty $.
    С точки зрения спектральных свойств этих форм, можно увидеть, что
    сферическая функция Бесселя
\begin{equation*}
    p_{1,\lambda}(r) = \frac{2r}{\sqrt{2\pi}\lambda^{2}}
	\frac{d}{dr}\frac{1}{r} \sin \lambda r ,
\end{equation*}
    которая входит в параметризацию несингулярного поперечного поля
(\ref{Atrns}),
    является предельным случаем функции
$ p_{1,\lambda}^{\kappa} $
\begin{equation*}
    p_{1,\lambda}(r)  
        = \lim_{\kappa\to\infty}
    \frac{2r}{\sqrt{2\pi}\lambda^{2}} \frac{d}{dr}\frac{1}{r}
    (\cos(\zeta +\lambda r) - \cos\zeta) ,\quad \zeta\to -\frac{\pi}{2}.
\end{equation*}

\subsection{Собственные состояния гамильтониана при 
$ l=1 $}
    Полученное в предыдущей части
    спектральное разложение позволяет теперь записать гауссов функционал
$ \phi_{0}^{\kappa}(u) $
    для расширенного квантового оператора
\begin{equation*}
    \HH_{1m}^{\kappa} = -\iint_{0}^{\infty} dr ds
    \frac{\delta}{\delta u(s)} T_{1}^{-1}(s,r)
	\frac{\delta}{\delta u(r)}
	+ \langle u, \check{T}_{\kappa}u\rangle_{1}
\end{equation*}
    как экспоненту от интегрального оператора
\begin{equation*}
    \phi_{0}^{\kappa}(u)
	= \exp\{-\frac{1}{2} \iint Q_{\kappa}^{\frac{1}{2}}(r,s)
	T_{r}u(r) T_{s}u(s) \, dr\,ds\} ,
\end{equation*}
    где
\begin{equation*}
    Q_{\kappa}^{\frac{1}{2}}(r,s) = \int p_{\lambda}(r) p_{\lambda}(s)
	\lambda\,d\lambda - i\kappa q(r)q(s) \bigr|_{\kappa <0} .
\end{equation*}
    В этом выражении мы специально вынесли дифференциальные операции
$ T_{r} $, 
$ T_{s} $
    для того, чтобы получить более гладкое ядро
$ Q_{\kappa}^{\frac{1}{2}} $.
    Несложно увидеть, что функционал
$ \phi_{0}^{\kappa} $
    удовлетворяет уравнению
\begin{equation*}
    \HH_{1m}^{\kappa} \phi_{0}^{\kappa}(u)
	= \Lambda_{0}^{\kappa} \phi_{0}^{\kappa}(u) ,\quad
    \Lambda_{0}^{\kappa} = \int_{0}^{\infty} T_{r}
	Q_{\kappa}^{\frac{1}{2}}(r,r') |_{r=r'} dr
\end{equation*}
    с некоторым бесконечным собственным значением
$ \Lambda_{0}^{\kappa} $.
    Операторы рождения и уничтожения с помощью ядра
$ Q_{\kappa}^{\frac{1}{2}} $
    записываются в виде
\begin{align*}
    b(r) &= \int_{0}^{\infty} \bigl( Q_{\kappa}^{\frac{1}{2}}(r,s)T_{s}u(s)
	- T_{1}^{-1}(r,s) \frac{\delta}{\delta u(r)} \bigr) ds , \\
    a(r) &= \frac{\delta}{\delta u(r)} + \int_{0}^{\infty} 
	T_{r} Q_{\kappa}^{\frac{1}{2}}(r,s) T_{s} u(s) ,
\end{align*}
    а инвариантные
$ n $-частичные состояния строятся с помощью формул, аналогичных
(\ref{Phin})
\begin{equation*}
    \phi_{\sigma_{n}}(u) = \iint
    \sigma_{n} (r_{1},\ldots r_{n})
	b(r_{1}) \ldots b(r_{n})
    dr_{1} \ldots dr_{n} \, \phi_{0}(u) .
\end{equation*}
    Для диагонализации оператора
$ \HH_{1m}^{\kappa} $
    перейдем к спектральному представлению квадратичной формы, то есть сделаем
    замену
\begin{equation*}
    \hat{u}(\lambda) = \int_{0}^{\infty} p_{\lambda}(r) T u(r)\,dr, \quad
	\hat{u}_{d} =  \int_{0}^{\infty} q(r) T u(r) \bigr|_{\kappa<0}
\end{equation*}
    (заметим, что все функции здесь вещественные), тогда
\begin{equation*}
    \HH_{1m}^{\kappa} = \int \bigl(
-\frac{\delta}{\delta \hat{u}(\lambda)} \frac{\delta}{\delta \hat{u}(\lambda)}
	+ \lambda^{2} \hat{u}^{2}(\lambda) \bigr)d\lambda
	- \kappa^{2} \hat{u}_{d}^{2} \bigr|_{\kappa <0} .
\end{equation*}
    Такому квантовому гамильтониану соответствуют операторы рождения
    и уничтожения
\begin{equation*}
    \hat{b}(\lambda) = \lambda \hat{u}(\lambda)
	- \frac{\delta}{\delta \hat{u}(\lambda)} ,\quad
    \hat{a}(\lambda) = \lambda \hat{u}(\lambda) 
	+ \frac{\delta}{\delta \hat{u}(\lambda)}
\end{equation*}
    и вакуумное состояние
\begin{equation*}
    \hat{\phi}_{0}(\hat{u}) = \phi_{0}(u(\hat{u})) = \exp\{-\frac{1}{2}
	\int_{0}^{\infty} \hat{u}^{2}(\lambda) \lambda \,d\lambda
	+\frac{i\kappa}{2} \hat{u}_{d}^{2} \bigr|_{\kappa < 0}\} .
\end{equation*}
    Собственные 
$ n $-частичные
    состояния строятся как интегралы c коэффициентами Бозе-Эйнштейна
$ \sigma(\lambda_{1},\ldots \sigma_{\lambda_{n}}) $
\begin{equation}
\label{hatphi}
    \hat{\phi}_{\sigma_{n}}(\hat{u}) = \iint
    \sigma_{n} (\lambda_{1},\ldots \lambda_{n}) \,
	\hat{b}(\lambda_{1}) \ldots \hat{b}(\lambda_{n}) \,
    d\lambda_{1} \ldots d\lambda_{n} \hat{\phi}_{0}(\hat{u}) ,
\end{equation}
    и, кроме того, при
$ \kappa < 0 $
    есть еще состояния, связанные с возбуждениями точечного спектра.

\subsection{Собственные состояния квантового гамильтониана свободного
поперечного поля}
    Собственные состояния квантового гамильтониана
(\ref{qH0}),
    в котором в место квадратичной формы
$ Q(A) $
    участвует ее расширение
(\ref{QkA}),
    строятся как произведения собственных состояний операторов
$ \HH_{lm}' $, 
$ 1\leq l, |m| \leq l $
$ \HH_{lm} $,
$ 2\leq l, |m| \leq l $
    и
$ \HH_{1m}^{\kappa} $.
    Для диагонализации первых двух наборов операторов можно использовать
    стандартное спектральное преобразование
\begin{equation*}
    \hat{u}_{lm}(\lambda)
	= \int_{0}^{\infty} p_{l,\lambda}(r) T_{l} u_{lm}(r)\,dr ,
    \quad \hat{w}_{lm}(\lambda)
	= \int_{0}^{\infty} \lambda p_{l,\lambda}(r) w_{lm}(r)\,dr ,
\end{equation*}
    где 
$ p_{l,\lambda}(r) $ --- это разновидность сферических функций Бесселя
\begin{equation*}
    p_{l,\lambda}(r) = \frac{2r^{l}}{\sqrt{2\pi}\lambda^{l+1}}
	\bigl(\frac{d}{dr}\frac{1}{r}\bigr)^{l} \sin \lambda r .
\end{equation*}
    Соответствующие операторы рождения и уничтожения, а также
    вакуумные состояния выглядят следующим образом
\begin{gather*}
    \hat{b}_{lm}(\lambda) = \lambda \hat{u}_{lm}(\lambda)
	- \frac{\delta}{\delta \hat{u}_{lm}(\lambda)} ,\quad
    \hat{a}_{lm}(\lambda) = \lambda \hat{u}_{lm}(\lambda) 
	+ \frac{\delta}{\delta \hat{u}_{lm}(\lambda)} \\
    \hat{b}'_{lm}(\lambda) = \lambda \hat{w}_{lm}(\lambda)
	- \frac{\delta}{\delta \hat{w}_{lm}(\lambda)} ,\quad
    \hat{a}'_{lm}(\lambda) = \lambda \hat{w}_{lm}(\lambda) 
	+ \frac{\delta}{\delta \hat{w}_{lm}(\lambda)} ,\\
    \hat{\phi}_{0}(\hat{u}_{lm}) = \exp\{-\frac{1}{2}
	\int_{0}^{\infty} \hat{u}_{lm}^{2}(\lambda) \lambda \,d\lambda \} ,\\
    \hat{\phi}'_{0}(\hat{w}_{lm}) = \exp\{-\frac{1}{2}
	\int_{0}^{\infty} \hat{w}_{lm}^{2}(\lambda) \lambda \,d\lambda \} .
\end{gather*}
    Диагонализация гамильтониана
$ \HH_{1m}^{\kappa} $
    с помощью преобразования
\begin{equation*}
    \hat{u}_{1m}(\lambda) = \int_{0}^{\infty} p_{1,\lambda}^{\kappa}(r)
	T_{1} u_{1m}(r)\,dr, 
\end{equation*}
    была описана в предыдущей части. Здесь стоит отметить, что в сферически
    несимметричном случае коэффициенты
$ \kappa $
    могут быть разными для компонент, соответствующим разным значениям
$ m $
    проекции момента вращения на третью ось координат.

    В результате, в переменных
$ \hat{u}_{lm} $,
$ \hat{w}_{lm} $
    получаем гамильтониан
\begin{equation*}
    \hat{\HH} = \sum_{-1\leq m\leq 1} \hat{\HH}_{1m}^{\kappa}
	+ \sum_{2\leq l, |m|\leq l} \hat{\HH}_{lm}
	+ \sum_{1\leq l, |m|\leq l} \hat{\HH}'_{lm} ,
\end{equation*}
    вакуумное состояние
\begin{equation*}
    \Phi_{0}^{\kappa} = \prod_{-1\leq m \leq 1} \phi_{1m}(\hat{u}_{1m}) \times
    \prod_{2\leq l, |m|\leq l} \phi_{lm}(\hat{u}_{lm}) \times
	\prod_{l,m} \phi'_{lm}(\hat{w}_{lm}) ,
\end{equation*}
    а
$ n $-частичные состояния получаются из формулы
(\ref{hatphi})
    с помощью замены оператора рождения
$ \hat{b}(\lambda) $ на
$ c(\lambda) $,
    который может принимать любое из значений
$ \hat{b}_{lm}(\lambda) $, $ \hat{b}'_{lm}(\lambda) $
\begin{equation*}
    \hat{\Phi}_{\sigma_{n}}(\hat{u}) = \iint
    \sigma_{n} (\lambda_{1},\ldots \lambda_{n}) \,
	c(\lambda_{1}) \ldots c(\lambda_{n}) \,
    d\lambda_{1} \ldots d\lambda_{n} \hat{\phi}_{0}(\hat{u}) .
\end{equation*}

\section{Заключение и обсуждения}
    Мы построили систему наборов состояний, удовлетворяющих собственным
    уравнениям для квантового оператора Гамильтона свободного поперечного
    поля.
    Полученные наборы в бщем случае
    зависят от выделенной точки пространства и не обладают
    масштабной инвариантностью (зависят от размерного параметра).
    Построение существенным образом использовало свойства расширений
    квадратичной формы оператора Лапласа, входящей в потенциальное слагаемое
    гамильтониана.
    Эти расширения могут быть записаны в ``инвариантной'' форме
(\ref{QkA}),
    которая, также как и условие поперечности, не подразумевает переход
    к сферическим координатам и использование какой-либо выделенной
    функциональной параметризации типа
(\ref{Atrexp}).
    Здесь возникает естественный вопрос о возможности обобщения формы
(\ref{QkA})
    на случай двух или нескольких выделенных точек пространства
\begin{equation*}
        Q_{\{\kappa\}}(A) = \lim_{r\to 0}\Bigl(
    \int_{\RR^{3}\setminus \{B_{r,n}\}}
        \bigl(\frac{\partial A_{k}}{\partial x_{j}}\bigr)^{2} d^{3} x -
    \sum_{n=1}^{N}\bigl(\frac{5}{3r}	+ \kappa_{n}\bigr)
	\int_{\partial B_{r,n}} |\vec{A}(\vec{x})|^{2} d^{2} s \Bigr) ,
\end{equation*}
    --- удовлетворяет ли такая форма условиям теоремы VIII.15 из
\cite{RS1},
    соответствует ли ей самосопряженный оператор, можно ли вычислить его
    спектральное представление. Существенная трудность при этом
    может заключаться в замыкании условия поперечности.
    Для случая одной особой точки мы просто предъявляем поперечное спектральное
    представление и далее можем рассматривать связанную с ним физику,
    а для случая нескольких точек такого представления может вообще
    не существовать.

    Другим важным замечанием является, является то, что по-видимому
    представление физического объекта (поля взаимодействия) в виде векторной
    функции на трехмерном пространстве не является правильным способом
    описания задачи. Две функции с сингулярностями в разных точках,
    которые могут являться представлениями одного и того же физического
    объекта в разные моменты времени, не выражаются через общий базис,
    то есть не имеют общего предстваления через один ортогональный набор.
    И, таким образом, имеется существенное препятствие в описании возможной
    динамики системы.

\section*{Благодарности}
    Работа выполнена при частичной поддержке грантов РФФИ 14-01-00341,
    15-01-03148 и программы ``Математические проблемы нелинейной динамики''
    РАН.

%\newpage
\begin{thebibliography}{0}

\bibitem{Dirac}
P.~A.~M.~Dirac,
``Quantum theory of emission and absorption of radiation,''
Proc.\ Roy.\ Soc.\ Lond.\ A {\bf 114} (1927) 243.

\bibitem{Becchi}
C.~M.~Becchi,
``Second quantization'', doi:10.4249/scholarpedia.7902,
http://www.scholarpedia.org/article/Second\_quantization

\bibitem{BF}
    F.~A.~Berezin, L.~D.~Faddeev,
  ``A Remark on Schrodinger's equation with a singular potential,''
  Sov.\ Math.\ Dokl.\  {\bf 2} (1961) 372
  [Dokl.\ Akad.\ Nauk Ser.\ Fiz.\  {\bf 137} (1961) 1011].

\bibitem{Jackiw}
  R.~Jackiw,
  ``Delta function potentials in two-dimensional and three-dimensional
  quantum mechanics,''
  In *Jackiw, R.: Diverse topics in theoretical and mathematical physics*
  35-53 (1991).

\bibitem{LFres}
L.~D.~Faddeev,
``Notes on divergences and dimensional transmutation in Yang-Mills theory,''
Theor.\ Math.\ Phys.\  {\bf 148} (2006) 986
[Teor.\ Mat.\ Fiz.\  {\bf 148} (2006) 133].

\bibitem{Fock}
V.~Fock, ``Konfigurationsraum und zweite Quantelung,''
Z. Phys. {\bf 75} (1932), 622-647.

\bibitem{FS}
  L.~D.~Faddeev and A.~A.~Slavnov,
\emph{Gauge Fields. Introduction To Quantum Theory},
Front.\ Phys.\  {\bf 50} (1980) 1, [Front.\ Phys.\  {\bf 83} (1990) 1].

\bibitem{Lapl} T.~A.~Bolokhov,
``Extensions of the quadratic form of the transverse Laplace operator'',
arXiv:1410.1487 [math.SP].

\bibitem{FStone}
    K.~Friedrichs, ``Spektraltheorie halbbeschr\"ankter Operatoren,''
    Math. Ann. {\bf 109}, 1934, 465--487;\\
    M.~Stone, in \emph{Linear Transformations in Hilbert spaces and their
    Applications in Analysis}, Amer. Math. Soc. Colloquim Publication {\bf 15},
    Providence, R.I., 1932;\\
    or see theorem X.23 in \cite{RS2}.

\bibitem{RS1}  M.~Reed, B.~Simon, \emph{Methods of Modern Mathematical
Physics. 1. Functional Analysis}, Academic Press New York London, 1972.

\bibitem{RS2} M.~Reed, B.~Simon, \emph{Methods of Modern Mathematical Physics.
II: Fourier Analysis, Self-adjointness}, Academic Press, 1975.

\bibitem{Inv} T.~A.~Bolokhov,
``Properties of the l=1 radial part of the Laplace operator in a special
scalar product'', arXiv:1510.07824 [math.SP].
    
\bibitem{VSH}
    B.~F.~Schutz, \emph{Geometrical methods of mathematical physics},
    Cambridge University Press, 1982;\\
    E.~L.~Hill, ``The Theory of Vector Spherical Harmonics'',
    Am. J. Phys. {\bf 22} (1954) 211.

%\bibitem{Krein}
%    M.~G.~Krein, ``The theory of self-adjoint extensions of semi-bounded
%Hermitian transformations and its applications.'',
%    Rec. Math. (Mat. Sbornik) N.S., {\bf 20} (62), 1947, 431--495;
%    Rec. Math. (Mat. Sbornik) N.S., {\bf 21} (64), 1947, 365--404.

%\bibitem{F}
%    Faddeev, Clay

%\bibitem{AK}
%    S.~Albeverio, P.~Kurasov, \emph{Singular Perturbation of Differential
%    Operators. Solvable Schr\"odinger type Operators},
%    Cambridge University Press, 2000.

%\bibitem{Richt}
%    R.~D.~Richtmyer, \emph{Principles of Advanced Mathematical Physics, vol.1},
%    Springer-Verlag, New York Heildelberg Berlin, 1978.
    

\end{thebibliography}

\end{document}

